\documentclass[10pt]{article}

\usepackage{enumerate}
\usepackage{amsfonts}

\usepackage[margin=1.5in,lmargin=1in,rmargin=1in]{geometry}

\begin{document}

\title{CS460 - Assignment 1}
\author{Matt Forbes}
\maketitle
\newpage

\section{Question 1.1}
Three main purposes of an OS:
\begin{enumerate}
\item Run programs
\item Interface with hardware
\item Store data and manage resources
\end{enumerate}

\section{Question 1.2}
Main differences between mainframes and PCs:
\begin{enumerate}
\item Ease of use vs. resource utilization
\item PC's should be responsize to the user
\item Mainframes should be efficient computation machines
\end{enumerate}

\section{Question 1.3}
Steps to run a program on dedicated machine:
\begin{enumerate}
\item Load bootstrap program from ROM
\item Initialize the system registers
\item Load the program in to memory
\item Run the program
\end{enumerate}

\section{Question 1.7}
Distinguishing kernel mode from user mode allows the use of privileged
instructions to be run in a controlled manner and not maliciously by a
user. Certain locations in memory could be controlled this way so that
ordinary users cannot mess with it.

\section{Question 1.8}
Should be privileged: a, c, e, f, g

\section{Question 1.13}
\begin{enumerate}[a)]
\item Two security problems:
  \begin{enumerate}[1]
  \item One user could possibly interfere with another user's processes
  \item One user could intercept device data and learn personal
    information about \\other users (credit cards, etc.)
  \end{enumerate}
\item Probably, it might cause the system to be less accessible due to
  security measures, though.
\end{enumerate}

\section{Question 1.17}
Symmetric processors all perform the same tasks within the OS, there
is no master-slave hierarchy. In asymmetric processing, one processor
is the master and delegates tasks to its slave processors. \\

\noindent Advantages:
\begin{enumerate}
\item Allows programs to be processed in parallel, which can result in
  massive speed-ups.
\item Makes it easier to run multiple programs concurrently with less
  context-switching.
\item Provides a way to increase computation power without optimizing
  processors themselves, but simply adding more to the system.
\end{enumerate}

\noindent Disadvantage: Programming easily parallelizable software can be a real
challenge and in some cases unintuitive.

\section{Question 1.23}
\begin{enumerate}[a)]
\item CPU interfaces with the device by: setting up buffers, pointers,
  and counters for the I/O device.
\item CPU knows when the operations are complete when the device
  driver generates an interrupt upon completion.
\item This process could interfere with user programs if said program
  also needs to utilize a busy device. It might seem as though the
  device is slow or broken if it can not be accessed how it normally
  would.
\end{enumerate}

\section{Question 1.30}
Definitions of types of operating systems:
\begin{enumerate}[a)]
\item Batch: Rather than a single user the computer at a time,
  multiple people queue programs to run and the OS does them each at a
  time. 
  
\item Interactive: Provides direct communication between the user and
  the system (i.e. using input devices such as a mouse and keyboard).
  
\item Time sharing: The CPU executes multiple jobs by switching among
  them, fast enough that it should be undetectable to the user.
  
\item Real time: A system is considered real time when there are
  certain time constraints that it must abide by. Known tasks must be
  consistently finished by an exact amount of time with no exceptions.
  
\item Network: Network operating systems provide features such as file
  sharing or communication across a network.
  
\item Parallel: Allows programs being run to be computed on multiple
  CPUs concurrently. 
  
\item Distributed: Collection of physically separate, possibly
  heterogenous, computer systems that are networked to provide the
  users with access to the various resources the system maintains.
  
\item Clustered: When multiple systems are somehow connected together
  to jointly provide computational power. Generally, the systems are
  connected together on a LAN for communication.
  
\item Handheld: Operating systems designed for use on devices small
  enough to be held in a hand. Usually an embedded system specifically
  designed to facilitate ease of use on a small screen with limited
  input capability.
  
\end{enumerate}


\end{document}
