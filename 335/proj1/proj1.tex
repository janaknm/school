\documentclass[a4paper,12pt]{article}
\usepackage{amsmath}
\usepackage{fullpage}
\usepackage{enumerate}

\begin{document}
\title{Project 1}
\author{Matt Forbes}
\date{November 8, 2010}
\maketitle

\begin{enumerate}[a)]
    \item L.P. Model for obtaining tablecloths at a minimum cost:

      \begin{itemize}
        \item
          There are three variables in the model. {\bf b}, which is how many
          new tablecloths are bought each day, where \(b_i\) denotes the number of tablecloths
          bought on day i. {\bf o} denotes how many cloths are sent to Bud's 'one-day'
          cleaning service, again \(o_i\) was how many are sent on day i. Finally,
          {\bf t} denotes the cloths sent to Mac's 'two-day' service.

        \item
          The objective function is pretty straightforward, each \(b_i\) costs \$8, \$10 to
          buy it and -\$2 when sold at the end of the period. Each towel sent to Bud, \(o_i\), costs \$3 and
          Mac, \(t_i\) costs \$1. 

        \item
          Each day has its own constraint that balances the available towels with how many are dirty, in
          cleaning, and in demand. 

        \item
          The next set of constraints are those that dictate the number of towels that can be sent to
          the cleaner each day, there are less of these than there are days.

        \item
          Finally, the last constraint is only applicable for 7+ day periods. It is responsible for not
          allowing use of Mac's service on specific days.

      \end{itemize}

      I'm not going to type out the actual LP problem because it is huge, but the above should give a rough
      outline of it. The result of solving this model for a 7day period was as follows:

      \begin{tabular}{r l}
        {\bf b}& = (110, 100, 160, 10, 0, 0, 0)\\
        {\bf o}& = (0, 0, 80, 120, 120)\\
        {\bf t}& = (110, 100, 80, 0)
      \end{tabular}

      This solution doesn't seem to fit any obvious pattern. The program buys a lot of towels at the beginning of
      the week and just cycles through them with around equal amounts of both services. 

    \item Model for the same problem with 30-day time period rather than one week:

      \begin{itemize}
        \item
          This is the exact same problem as a) just with more variables, and the constraint that Mac's cleaning
          service does not accept cloths on the weekends. The problem statement does not specify that we can not
          pick up cloths on the weekend, so this assumes that we can. 

        \item
          In the variable descriptions above, it was noted that the last constraint in the constraint matrix
          is what controls whether or not we can send cloths to Mac on certain days. All the variables are restricted
          to being non-negative, so the last constraint says \(0 = o_i + \dots + o_j\) for some i's, j's, and indices
          between. Weekend \(o_i\)s are those which are indexed at a multiple of 6 or 7 
          relative to the first monday. The \(o_i\)s that correspond to weekend days are set to 1, 
          so that the only way that constraint is met is if all of those \(o_i\)s are equal to 0. \(o_i\)s that are
          set to 0 mean that no cloths were sent to Mac that day, which was the goal of the constraint.

      \end{itemize}

      The solution to this new problem is as follows:

      \begin{tabular}{r l}
        {\bf b}& = (110, 100, 160, 120, 0, 10, 0, 0, \dots , 0)\\
        {\bf o}& = (0, 0, \dots , 0)\\
        {\bf t}& = (0, 180, 190, 120, 110, 100, 160, 120, 180, 200, 120, 110, 100, 160, \\ 
        {}& \quad 120, 180, 200, 120, 110, 100, 160, 120, 180, 200, 120, 110, 100)
      \end{tabular}

      This solution makes a lot of sense, the program just stocks up on towels to start with and only uses the
      two-day service. Only using the two-day service meant the program had to buy more towels to start with, but
      after that was able to time the cleanings perfectly such that it never had to buy more towels or use the
      expensive one-day service.

    \item 30-day period with one-day service only \$1.50:

      Solution:

      \begin{tabular}{r l}
        {\bf b}& = (110, 100, 160, 60, 0, 0, \dots , 0)\\
        {\bf o}& = (0, 0, 30, 70, 70, 0, 0, 0, 0, 30, 70, 70, 0, 0, 0, 0, 30, 70, 70, \\
        {}& \quad 0, 0, 0, 0, 30, 70, 70, 0, 0)\\
        {\bf t}& = (60, 150, 130, 50, 110, 100, 160, 120, 120, 150, 130, 50, 110, 100, \\
        {}& \quad 160, 120, 150, 130, 50, 110, 100, 160, 120, 150, 130, 50, 110, 100)\\
      \end{tabular}

      In this solution, we actually take advantage of the expedient one-day service now that the
      price is reasonable. The major tactic is still to use the cheaper two-day service, but in this
      case it is cheaper to use some one-day cleanings before really crazy days.

\end{enumerate}

\end{document}
