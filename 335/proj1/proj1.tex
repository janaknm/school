\documentclass[a4paper,12pt]{article}
\usepackage{amsmath}
\usepackage{fullpage}
\usepackage{enumerate}

\begin{document}
\title{Project 1}
\author{Matt Forbes}
\date{November 8, 2010}
\maketitle

\begin{enumerate}[a)]
    \item L.P. Model for obtaining tablecloths at a minimum cost:

      \begin{itemize}
        \item
          There are four variables in the model. {\bf b}, which is how many
          new tablecloths are bought each day, where \(b_i\) denotes the number of tablecloths
          bought on day i. {\bf l} is the number of leftover (dirty) towels
          at the end of the day. {\bf o} denotes how many cloths are sent to Bud's 'one-day'
          cleaning service, again \(o_i\) was how many are sent on day i. Finally,
          {\bf t} denotes the cloths sent to Mac's 'two-day' service.

        \item
          The objective function is pretty straightforward, each \(b_i\) costs \$8, \$10 to
          buy it and -\$2 when sold at the end of the period. Each towel sent to Bud, \(o_i\), costs \$3 and
          Mac, \(t_i\) costs \$1. 

        \item
          For each day in the period there are two constraints. One is the relationship of how many towels need
          to be bought for that day, which is the number needed minus the clean towels coming in. Second, a
          constraint needs to be placed on how many towels can be washed (which can't be greater than how many
          are dirty).

        \item
          Finally, the last constraint is only applicable for 7+ day periods. It is responsible for not
          allowing use of Mac's service on specific days.

      \end{itemize}

      The solution is at the end of the Mathematica code, labeled as '7 days, normal pricing.' There isn't a
      really obvious pattern to this program which didn't favor Mac or Bud's service over the other.

    \item Model for the same problem with 30-day time period rather than one week:

      \begin{itemize}
        \item
          This is the exact same problem as a) just with more variables, and the constraint that Mac's cleaning
          service does not accept cloths on the weekends. The problem statement does not specify that we can not
          pick up cloths on the weekend, so this assumes that we can. 

        \item
          In the variable descriptions above, it was noted that the last constraint in the constraint matrix
          is what controls whether or not we can send cloths to Mac on certain days. All the variables are restricted
          to being non-negative, so the last constraint says \(0 = o_i + \dots + o_j\) for some i's, j's, and indices
          between. Weekend \(o_i\)s are those which are indexed at a multiple of 6 or 7 
          relative to the first monday. The \(o_i\)s that correspond to weekend days are set to 1, 
          so that the only way that constraint is met is if all of those \(o_i\)s are equal to 0. \(o_i\)s that are
          set to 0 mean that no cloths were sent to Mac that day, which was the goal of the constraint.

      \end{itemize}

        There is a very visible pattern to this solution. Towel purchases were high to start with and then only Mac's
        cheaper service was used except for on the weekends where it wasn't open. It isn't too much of a stretch to 
        see that a good solution would prioritize the cheaper service, and that is what this does.

    \item 30-day period with one-day service only \$1.50:

      In this solution, we actually take advantage of the expedient one-day service now that the
      price is reasonable. The major tactic is still to use the cheaper two-day service, but in this
      case it is cheaper to use some one-day cleanings before really crazy days.

\end{enumerate}

\end{document}
