%        File: assg1.tex
%     Created: Wed Oct 06 08:00 PM 2010 P
% Last Change: Wed Oct 06 08:00 PM 2010 P
%
\documentclass[a4paper,12pt]{article}
\usepackage{amsmath}
\begin{document}
\noindent Matt Forbes \\
October 7 \\
Homework 1 \\


\section{Problem One}
\subsection{}
The minimum number of socks that need to be drawn to guarantee a pair of any color is 4.
You could possibly have 3 socks that do not make a pair, but on the next draw, one of the 
socks will become a match.
\subsection{}
Likewise, the minimum number of socks from a drawer of n different color socks (assuming there is at least 2 of each color)
is n+1. For the same reason that you can have no match on the nth draw, but on the n+1th draw, one sock will match.

\section{Problem Two}
Prove by induction that \(F_{n+k} = F_kF_{n+1} + F_{k-1}F_n\)
\subsection{Basis}
\begin{align*}
  \text{let n=0, k \(>=\) 1} \\
  F_{0+k} = F_kF_1 + F_{k-1}F_0 \\
  F_k = F_k(1) + F_{k-1}(0) \\
  F_k = F_k
\end{align*}

\subsection{Inductive Hypothesis}
For all \(n >= 1, c >= 0 \), assuming the equation is true for \(n-1\) then it 
is also true for \(n\).

\subsection{Proof}
\begin{eqnarray*}
  F_{n+c} &=& F_cF_{n+1} + F_{c-1}F_n \\
  F_{(n-1) + (c+1)} &=& F_cF_{n+1} + F_{c-1}F_n \\
  F_{c+1}F_n + F_cF_{n-1} &=& F_c(F_{n-1} + F_n) + F_{c-1}F_n \\
  (F_{c-1} + F_c)F_n + F_cF_{n-1} &=& F_cF_{n-1} + F_cF_n + F_{c-1}F_n \\
  F_{c-1}F_n + F_cF_n + F_cF_{n-1} &=& F_{c-1}F_n + F_cF_n + F_cF_{n-1} \\
\end{eqnarray*}

\section{Problem Three}
Prove \(\sum_{i=0}^{n} \dbinom{n}{i} = 2^n\) with induction and given identity.
\subsection{Basis}
\begin{eqnarray*}
  \sum_{i=0}^{1} \dbinom{1}{i} &=& \dbinom{1}{0} + \dbinom{1}{1} \\
  &=& 1 + 1 \\
  &=& 2^1 \\
\end{eqnarray*}
\subsection{Inductive Hypothesis}
For all \(n > 1\), assuming the sum is true for \(n-1\), it is also true for \(n\).
\subsection{Induction}
\begin{eqnarray*}
  \sum_{i=0}^{n} \dbinom{n}{i} &=& \dbinom{n}{0} + \sum_{i=1}^{n} \dbinom{n}{i} \text{[pull first value from sum]} \\
  &=& 1 + \sum_{i=1}^{n} \dbinom{n-1}{i} + \dbinom{n-1}{i-1} \text{[given identity]} \\
  &=& 1 + \sum_{i=1}^{n} \dbinom{n-1}{i} + \sum_{i=1}^{n} \dbinom{n-1}{i-1} \text{[break up sum]} \\
  &=& 1 + [(\sum_{i=0}^{n} \dbinom{n-1}{i}) - \dbinom{n}{0}] + \sum_{i=0}^{n} \dbinom{n-1}{i} \text{[change sum limits]} \\
  &=& 1 + [2^{n-1} - 1] + 2^{n-1} \text{[inductive hypthosis]} \\
  &=& 2*2^{n-1} \\
  &=& 2^n
\end{eqnarray*}
\section{Problem Four}
\subsection{Measure 1: Simplicity of Plan}
I had a hard time quantifying the efficiencies of this problem, but I think that since we are working
with people, not computers, the ease of carrying out the plan can be a large factor. A complex plan might
end up bringing everyone together faster, but if it is too complicated, some of the troops may screw it
up. So this plan is optimized to provide easy instructions to the troops. \\

\noindent n = the number of troops in the team.\\
\(x_i\) = soldier i, where i = 1...n \\
\(d_i\) = distance assigned to \(x_i\), where i = 1...n \\
let \(d_i\) = \((-1)^{i}i\), where i = 1...n \\

\noindent {\bf Walking Plan: (for \(x_i\))}
\begin{enumerate}
\item Walk \(d_i\) meters (where -d is to the left).
\item If you reach another soldier, then stay together as a group, choosing your new \(d\) to be
the greatest \(d\) of the group.
\item Your new \(d\) is now \(-2d\)
\item If you do not have all members of the team in your group, do step 1 again. 
\end{enumerate}

\noindent {\bf Why this works} \\
This will always work because the soldier with the largest initial \(d\) will end crossing the path
of all the other soldiers at one point because he will be going the furthest in both directions. The
other soldiers are also moving, and will hopefully group up faster rather than just waiting to be
picked up by the furthest mover.

\subsection{Measure 2: Fastest Rendezvous of Team}
The obvious optimization is the time it takes to get the whole group together in the shortest
amount of time. This plan is quite a bit more complex, but it should speed up the grouping by quite
a bit. It uses a similar algorithm as the first plan, but handles the collision of two soldiers quite a 
bit differently. \\

\noindent n = the number of troops in the team.\\
\(x_i\) = soldier i, where i = 1...n \\
\(d_i\) = distance assigned to \(x_i\), where i = 1...n \\
let \(d_i\) = \((-1)^{i}i\), where i = 1...n \\


\noindent {\bf Walking Plan: (for \(x_i\))} \\
\underline{Case 1 (Default):}
\begin{enumerate}
\item Walk \(d_i\) meters (where -d is to the left).
\item If you meet another soldier, you are now a group. Both of your \(d\) values are now equal to the
greatest of the two, or to your teammates value if he is in a group. You  also keep your original direction.
Your new goal is to find all the soldiers on your side of where you met. Go to case 2.
\item If you didn't meet another soldier, then your \(d\) is now \(-2d\).
\item Go to case 1.
\end{enumerate}
\underline{Case 2 (Grouped with one partner):}
\begin{enumerate}
\item You and your "partner" are now responsible for reporting the number of teammates you each find on the
side of the meeting point until you have found the entire team.
\item Walk \(d\) meters (where -d is to the left).
\item If you meet another soldier, you have completed a group. And your new job is simply to relay the number
of teammates currently found on your side of the field. Go to case 3.
\item If you didn't meet another solder, then walk \(-d\) meters back to where you first collided with
a member, and meet them there.
\item Add up the total number of soldiers found on both sides, and if all members are found then go to case 4.
\item Otherwise, your \(d\) is now \(2d\), go to case 2.2.
\end{enumerate}
\underline{Case 3 (Grouped on both sides):}
\begin{enumerate}
\item Walk \(-d\) meters back where you had your first collision, and meet your "partner" there.
\item Add up the total number of soldiers on both sides (including soldiers your new teammate found), and if
all members are accounted for go to case 4.
\item Otherwise, your \(d\) is now \(2d\).
\item Walk \(d\) meters.
\item go to case 3.
\end{enumerate}
\underline{Case 4 (Final):}
\begin{enumerate}
\item If there are no teammates in the positive direction, then stop and wait for the group to come to you.
\item Otherwise, if there are no teammates in the negative direction, just walk in the positive direction
until you meet up with the rest of the team.
\item Otherwise, walk \(d\) meters and tell them that everyone is accounted for, then walk in the positve
direction until you meet up with the rest of the team.
\end{enumerate}

\noindent {\bf Why this works} \\
This is a similar routine as the first plan. The only thing is that rather than alternating directions
after you find a teammate, you are smart about it and split the problem in half. You look on one side, and 
they look on the other. This will always work, because everyone starts by probing for nearby members, then
scouts on one side of that collision. By moving further and further in that direction everytime, they are 
guaranteed to find everyone else on that side of them.

\end{document}


