\documentclass[a4paper,12pt]{article}
\usepackage{amsmath}
\usepackage{enumerate}

\pagestyle{empty} \setlength{\parindent}{0mm}
\addtolength{\topmargin}{-0.5in} \setlength{\textheight}{9in}
\addtolength{\textwidth}{1in} \addtolength{\oddsidemargin}{-0.5in}


\begin{document}
\title{Homework 4}
\author{Matt Forbes}
\date{November 16, 2010}
\maketitle

\section*{Problem 1}


\begin{enumerate}[a)]
  \item 

    Using a linked list is probably the most straight-forward way to
    simulate eeny meeny miny mo. To build the list, we would start
    with the first kid as the head of the list. Then we append each
    kid to the list in a row, going around the circle to the right.

    \noindent Once, we have the list built, simulating the game would
    just be moving forward in the list for each word in the song, and
    removing the element we hit when the song is finished. So,
    starting with the head of the list, visit the next node w times
    (wrapping around to the head when we hit the end of the
    list). After w visits, we remove the current element we 
    visiting from the list.

    \noindent We do this while the number of elements in the list is
    greater than 1. When there is only one kid left in the list, he is
    'it.'

    \noindent This way, we are doing w visits for each iteration of
    the song (definition of the game). And we have to remove p-1 kids
    from the list (p-1 iterations), so in total, we are visiting
    about \(pw\) nodes in the list.
    
    \noindent So in total, we did p insertions to the list, and pw
    visits on the list. Therefore, the running time of the algorithm
    is \(O(p + pw)\) or just \(O(pw)\).


  \item
    
\subsection*{Assumptions}

    This code assumes that we have the augmented data structure
    OS\_Tree that we described and used in class, with methods
    OS\_Insert(T, obj) and OS\_Select(T, r) which selects the node
    within tree T with rank r. This OS\_Tree holds kid datatypes that
    hold a pointer to the kid and also their original index in the
    circle, labeled kid and index respectively. The tree sorts on
    index.
    
\pagebreak

\subsection*{Pseduo Code}

    \begin{verbatim}
      (* emmm_select performs eeny-meeny-miny-mo selection on the 
      array of p kids (kid_arr), using a song with w words *)

      def emmm_select(kid_arr, p, w):
          OS_Tree T = new OS_Tree
          
          (* insert kids in original order *)
          for i in 1...p:
              OS_Insert(T, { kid_arr[i], i } )

          (* simulate p-1 iterations of the song *)
          for i in 1...(p-1):
              rank = w % (p-i+1) (* rank of kid to be removed *)
              kid = OS_Select(T, rank)
              remove(T, kid)

          (* return last kid left in T *)
          return OS_Select(T, 1).kid

    \end{verbatim}

\subsection*{Justification}

This algorithm works by first creating an order-statistic balanced
binary tree (the one from class), and filling it with the kids playing
the game. The ranks of the kids are determined by their distance from
the current song-singer (whose rank is 1). So the singer's rank is 1,
the kid to his right is rank 2, all the way around the circle. The kid
who is going to be removed from the circle is the one who is w kids to
the right of the singer (with wrapping). This rank is just w \(\%\) number
of kids left in the circle. Determing this rank and retrieving the kid
from the tree is \(O(\log_2(p))\) using OS\_Select. We need to remove
\(p-1\) kids, so the algorithm's running time is \(O(p\log_2(p))\).
     

\end{enumerate}

\pagebreak

\section*{Problem 2}

\subsection*{Insert Pseudocode}


\begin{verbatim}

  (* assumed definitions: *)
  (* add_to_end - create p-heap node at end of H with given items and return it *)
  (* get_parent - return parent of given node *)
  (* get_elem - get ith element out of a node *)
  (* swap(node,i,node2,j)  - swap ith element of node with jth element of node2 *)


  (* insert elems (array length p) in to p-heap H *)
  def insert(H, p, elems):
      node = add_to_end(H, elems)
      for i in 1...p:
          bubble_up(node, i, p)

  def bubble_up(node, i, p):
      parent = get_parent(node)
      elem = get_elem(node, i)
      best = infinity
      best_ix = nil
      for j in 1...p:
          parent_elem = get_elem(parent, j)
          if elem > parent_elem and parent_elem < best:
              best = parent_elem
              best_ix = j
      if best_ix is not nil:
          swap(node, i, parent, best_ix)
          bubble_up(parent, best_ix, p)
          
\end{verbatim}

\pagebreak

\subsection*{Extract Max Pseudocode}

\begin{verbatim}

  (* assumed definitions: *)
  (* get_elem - get ith element out of a node *)
  (* swap(node,i,node2,j)  - swap ith element of node with jth element of node2 *)
  (* left/right - get left or right subtree of a node *)
  (* detach_root - remove root node from heap and return it *)
  (* attach_root - set root of the tree to given node *)
  (* detach_last - remove last node from heap and return it *)
  (* items - return items on a node *)

  (* remove p greatest elements in heap *)
  def extract_max(H, p):
      top = detach_root(H)
      last = detach_last(H)
      attach_root(H, last)
      for i in 1...p:
          bubble_down(last, i)
      return items(top)

  def bubble_down(node, i, p):
      best = get_element(node, i)
      best_ix = nil
      best_node = nil
      for j in 1...p:
          check = get_element(left(node), j)
          if check > best:
              best = check
              best_ix = j
              best_node = left(node)
          check = get_element(right(node), j)
          if check > best:
              best = check
              best_ix = j
              best_node = right(node)
      if best_ix is not nil:
          swap(node, i, best_node, best_ix)
          bubble_down(best_node, best_ix, p)

      

\end{verbatim}



\end{document}
