%        File: assg1.tex
%     Created: Wed Oct 06 08:00 PM 2010 P
% Last Change: Wed Oct 06 08:00 PM 2010 P
%
\documentclass[a4paper,12pt]{article}
\usepackage{amsmath}
\usepackage{fullpage}
\begin{document}
\title{Homework 2}
\author{Matt Forbes}
\date{October 21, 2010}
\maketitle

\newenvironment{indentpar}[1]%
{\begin{list}{}%
         {\setlength{\leftmargin}{#1}}%
         \item[]%
}
{\end{list}}

\section{Problem One}
\subsection*{a\()\)}
{\bf found} is true. For this case to occur, we had to be within some iteration of the loop. In order to be in
an interation, the expression \(\text{(low < high) AND !found}\) must be true. Therefore, low does not equal
high, because low must be less than high. The first if statement within the loop body checks if\\
\(A[low] + a[high] ==  x\) is true, and will set found to true if that is the case. Therefore, if found is true,
then low and high are indices of A whose elements' sum is x.
\subsection*{b\()\)}
{\bf low \(\ge\) high}. I will prove the loop invariant provided to show that if low \(\ge\) high, 
there does not exist two disting elements in {\bf A} that sum to x.\\\\
\underline{Basis}
\begin{indentpar}{1cm}
  At the start of the first iteration,\\
  \(low = 0, high = n-1,\\ 
  S = \{A[0]\dots A[-1]\} \cup \{A[n] \dots A[n-1]\} = \{\}\)\\
  So there are no distinct pairs in S sum to x. The L.I. holds before we enter the loop.
\end{indentpar}
\underline{Maintenance}
\begin{indentpar}{1cm}
  Assuming the L.I. held for all iterations up to this iteration j, then:
  \begin{itemize}
    \item Right before this loop started, there was no distinct pair of elements in the set 
      \(S = \{A[0] \dots A[low-1]\} \cup \{A[high+1] \dots A[n-1]\}\) whose sum = x.
    \item During this iteration, \(A[low] + a[high]\) could be:\\
      {\bf equal to x}: We found a distinct pair that sums x, done.\\\\
      {\bf less than x}: low is incremented, and thus \(A[low]\) is 'added' to S in the next iteration, in which
      case the L.I. would still hold for these reasons: \(A[low] + A[i], i= 0 \dots low-1\) will always be less than
      x, and \(A[low] + a[j], j = high+1 \dots n-1\) will always be greater than x. Therefore there will be no pair in S
      whose sum is exactly x.\\\\
      {\bf greater than x}: high is decremented, and this \(A[high]\) is 'added' to S in the next iteration, in 
      which case the L.I. would still hold for these reasons: \(A[high] + A[j], j = high+1 \dots n-1\) will always
      be greater than x, and \(A[i] + A[high], i = 0 \dots low\) will always be less than x. Therefore there will be
      no pair in S whose sum is exactly x.
    \item The L.I. will always hold after this iteration, granted that it held up to this point for the reasons listed
      above.
  \end{itemize}
\underline{Termination}
\begin{indentpar}{1cm}
  \begin{itemize}
    \item After each iteration, either low is incremented or high is decremented, so they have to converge at one
      point as long as {\bf found} is never set to true, so the loop is guaranteed to terminate.
    \item According to the L.I. at the end of the last iteration, there is no distinct pair of elements in 
      \(S = \{A[0] \dots A[low-1] \} \cup \{A[high+1] \dots A[n-1]\}\) whose sum is x. Well at the end of the loop,
      S is equal to all of the elements in {\bf A}. Therefore, there is no distinct pair of elements in {\bf A} whose
      sum is equal to x.
  \end{itemize}
\end{indentpar}
\end{indentpar}
\section{Problem One}


\end{document}


