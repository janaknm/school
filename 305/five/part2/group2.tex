\documentclass[a4paper,12pt]{article}
\usepackage{amsmath}
\usepackage{fullpage}
\usepackage{color}
\usepackage{enumerate}

\begin{document}
\title{Five or More Group Analysis\\ Part Two}
\date{October 21, 2010}
\author{Matt Forbes \\ Nick Fitzgerald \\ KC Faulkner \\ Chris LOL}
\maketitle

\section*{Definitions}
\begin{tabular}{r l}
    Queue:& List of marbles dropping on the board in the next turn. \\
    Group:& A set of same-color marbles on the board that are in an\\
    {}& unobstructed row of 5, but not necessarily touching. \\
    Attach:& Moving a marble to a group's edge.\\
    Completed:& When a group gets five marbles in a row and is taken off the board.\\
    k-Connected:& There are k acyclical paths between a marble and an associated group.\\
    Blocked:& There are no paths between this marble and an associated group.\\
    Region:& Continous unoccupied area on the board. 
    
\end{tabular}

\section*{Group Selection}

Ranked in order of highest priority. When multiple color groups match a rule, the use
the list of tie breaking rules, or continue to the next rule if no rules break the tie.

\begin{enumerate}[1)]

    \item Choose a group that has the possibility of being completed with one available move.

    \item Group with the most marbles in it. Break ties by:
        \begin{enumerate}[a)]
            \item Has the most same-colored connected marbles on the board.
            \item Has the most same-colored marbles in the queue. 
        \end{enumerate}

    \item If no groups have conected marbles, then we should start a new group, rules for which
        color and where to start it are discussed below.

\end{enumerate}

\section*{Actual Marble Selection}

After deciding which group to move, these are the rules on how to pick which marble should be moved to the group
if there are multiple options. These rules also discuss which side of the group the marble should be attached to.

\begin{enumerate}[1)]
    
    \item Pick the marble that would open up the most paths on the board when moved. 
    \item Prioritize marbles that open up paths to other large groups.
    \item Pick marbles that have small k-connections. They are more
      likely to be blocked by incoming marbles from the queue.

\end{enumerate}

\section*{How to Move a Selected Marble}

\begin{enumerate}[1)]
    \item Attach marbles to a space in a  group that has the least
      paths to it. That position is more likely to become blocked by the
      incoming marbles.
    \item Don't attach a marble to the side of a group if that move
      would divide a region (unless that move will complete the group).
      
\end{enumerate}

\section*{Starting a group}

These are guidelines about when no groups are ideal or available, so we need to start a new one.

\begin{enumerate}[1)]
    
    \item Pick the color that has the most same-color connected
      marbles on the board that has sufficient room to be completed.
    \item Pick a base marble which has the most connected marbles of
      the same color and  has sufficient room to complete the
      group. Then pick the marble with smallest k-connection to move
      to it. This marble is the most likely to be blocked after the
      incoming marbles are dropped.
    \item Avoid starting diagonal groups. The rectangular area
      occupied by a diagonal group is completely unavailable to other
      horizontal and vertical groups. While a vertical or horizontal
      group occupies at most 5 squares, a diagonal group occupies 25.

\end{enumerate}

\end{document}
