\documentclass[12pt]{article}
\usepackage{amsmath}
\usepackage{amsfonts}
\usepackage{fancyhdr}
\usepackage[margin=1in,lmargin=1in,rmargin=1in]{geometry}

\def \TITLE {Tree Leaf Allometry}

%Math comp id: 13762
\def \GROUPID {13762}

\pagestyle{fancy}

\fancyhead{}
\fancyfoot{}

\fancyhead[CO,CE]{\TITLE}
\fancyfoot[CO,CE]{Mathematical Contest in Modeling}
\renewcommand{\footrulewidth}{0.4pt}


\lhead{\GROUPID}
\rhead{Summary}

\begin{document}


Topics of global warming, solar energy, and academia can be further
explored by investigating the quantification of the leaf mass of trees
and its relation to energy absorption and efficiency. We present a
model which showcases these properties in a theoretical manner.  Using
convex, continuous, differentiable curves in $\mathbb{R}^3$, we model
trees as 3-dimensional surfaces produced by rotating said curves about
the z-axis. In our model, trees grow in the positive z direction and
lie directly underneath the path of the Sun. At midday, the Sun will
be precisely above the center of the tree.

We frequently use the term ``leaf density'' to denote the number of
leaves per unit volume. It is important to distinguish this from
``leaf mass density'' which is the usual mass per unit volume. It is
our assumption that the maximum leaf density for a given height,
$z_0$, is directly proportional to the daily energy observed along the
tree's profile at the same height $z_0$. We claim that knowing the
coefficient of proportionality, $\alpha$, of these two quantities and
$\gamma$, the leaf mass density, we can estimate the total leaf mass
of a given tree.

In the process of determining leaf mass, we provide an expression for
leaf density, which we call $\rho$. Based on this $\rho$, we define a
probability density for leaf stem locations as a function of distance
from the tree's trunk. We compare the relative efficiency of leaf
shapes by running a probabilistic simulation which plots leaves on a
branch according to the aforementioned distribution.

Our results include a symbolic expression for leaf mass with respect
to the tree's profile. Further, we show that leaves with natural,
leaf-like (folium) shapes perform much better than non-traditional
leaf shapes such as squares and circles. It is interesting to note
that circles and squares are conventionally optimal shapes (such as in
coverage and packing problems), but are suboptimal in the case of
leaves. Performance was based on the percentage of overlapping leaf
surface area across a branch.

We provide many theoretical relationships between model variables
which are strengthened by experimental data produced by others. We
developed relationships rather than specific calculations, providing a
more conceptual solution to the problem.



\end{document}
