\documentclass[10pt,twocolumn]{article}
\usepackage[pdftex]{graphicx}
\usepackage{subfig}
\usepackage{amsfonts}
\usepackage{amsmath}
\usepackage{calc}
\usepackage{lipsum}

\newlength{\imgwidth}

\newcommand\scalegraphics[1]{%   
    \settowidth{\imgwidth}{\includegraphics{#1}}%
    \setlength{\imgwidth}{\minof{\imgwidth}{\columnwidth}}%
    \includegraphics[width=\imgwidth,keepaspectratio]{#1}%
}

\pagestyle{empty} \setlength{\parindent}{0mm}
\addtolength{\topmargin}{-0.5in} \setlength{\textheight}{9in}
\addtolength{\textwidth}{1in} \addtolength{\oddsidemargin}{-0.5in}

\begin{document}

\section{Model}

We model a tree by taking some function $x = l(z)$ and rotating it
about the $z$ axis. The resulting three-dimensional surface is the
profile of some tree $L$. We require $l(z)$ to be both convex and
continuous for $z_0 \le z \le z_1$ where $z_0$ and $z_1$ are the lower
and upper bounds for the tree. \\

\begin{figure}[h!]
  \caption{$l(z)=\sqrt{1-(\frac{z}{2})^2}$}
  \centering
  \scalegraphics{img/lz.png}
  \label{fig:l(z)}
\end{figure}

For example, we take $l(z) = \sqrt{1-(\frac{z}{2})^2}$, rotate it around
the z axis giving a surface which represents the profile of a tree
(see {\bf figures~\ref{fig:l(z)} and~\ref{fig:L}}). Describing tree
profiles in this fashion is not only convenient, but fairly
representative of trees in nature [CITATION ABOUT TREE SYMMETRY]. \\

\begin{figure}[h!]
  \centering
  \scalegraphics{img/L.png}
  \caption{Surface $L$}
  \label{fig:L}
\end{figure}

Having a model of trees in $\mathbb{R}^3$, we now wish to represent
incoming sunlight in relation to our tree $L$. To simplify this
relationship, we assume the Sun's path coincides with $y=0$; in other
words, the Sun travels directly over the tree along the $x$ axis. We
choose the x-y plane as being parallel to the Earth's surface, and let
$\theta_t$ denote the angle rays from the Sun make with the positive
$x$ axis at time $t$. Letting $t$ range from $0$ to $1$ we have
$\theta_t = \theta_{min} + t(\theta_{max} - \theta_{min})$ where
$\theta_{min}$ and $\theta_{max}$ are the minimum and maximum angles
for which Sun rays will reach $L$ respectively. \\

We let $\vec{T_z}$ denote the tangent vector at a point \\ $P$ =
($l(z)$, 0, $z$) on $L$. The intensity vector $\vec{I_\theta}$
represents a Sun ray that makes an angle $\theta$ with the x-y
plane. $|\vec{I_\theta}| = 1367$ for all $\theta$ [CITATION FOR SUN
  INTENSITY]. {\bf Figure~\ref{fig:vectors}} shows the previously defined
vectors and angles.

\begin{figure}[h!]
  \centering
  \caption{Angles and vectors on $l(z)$}
  \scalegraphics{img/vectors.png}
  \label{fig:vectors}
\end{figure}

\subsection{Finding Energy at a Point}
We now wish to determine the total energy a point receives over one
period (a full day). First, we examine the instantaneous intensity at
point $P$ = ($l(z)$, 0, $z$) using [CITATION FOR EQN]:
\begin{center}
  \begin{equation}\label{eqn:intbasic}
    |I_z| = |I_\theta| \  cos\phi
  \end{equation}
\end{center}

Here, $\phi$ is the angle between $\vec{I_\theta}^{\perp}$ and
$\vec{T_z}$, notice this is just the projection of $\vec{I_\theta}$ on
to $\vec{T_z}$. Solving for $\phi$ using the definition of the dot
product yields:

\begin{center}
  \begin{gather}
    \phi = cos^{-1} \left(\dfrac{\vec{I_0}^{\perp} \cdot \vec{T_z}}{|\vec{I_0}^{\perp}| |\vec{T_z}|}\right) \\
    I(z,\theta) = |I_\theta|\left(\dfrac{-sin\theta \  l'(z) + cos\theta}{\sqrt{1+(l'(z))^2}}\right)\label{eqn:intpoint}
  \end{gather}
\end{center}

Equation~\eqref{eqn:intpoint} defines the instantaneous intensity at a
point ($l(z)$, 0, $z$) for a given $\theta$. Having an expression for
intensity allows us to determine the total energy a point receives
over the course of one full day. To calculate total energy, we must
integrate intensity over a full period, $0 \le t \le 1$ [CITATION].

\begin{center}
  \begin{equation}
    E(z) = \int_0^1 I(z,\theta_t) dt
  \end{equation}
\end{center}

Continuing with our hypothetical tree with profile $l(z) =
\sqrt{1-(\frac{z}{2})^2}$, we graph the energy observed per day for
each z for which our tree is defined ($-3 \le z \le 3$). This graph
({\bf figure~\ref{fig:energy}}) shows exactly what you might expect:
very small energy at the base of the tree, fairly average energy in
the middle, and very high energy near the top.

\begin{figure}[h!]
  \centering
  \scalegraphics{img/energy.png}
  \caption{Energy over a full day with respect to z}
  \label{fig:energy}
\end{figure}

\subsection{Estimating Leaf Mass}
Our approach to estimating the leaf mass requires knowledge of the
leaf density, $\rho$, as a function of height and lateral distance
from the trunk. It is our assumption that the tree is fully
symmetrical about the $z$ axis, thus we can simply work in the
[[slice]] $y=0$. \\

We believe that the leaf density is logistic with respect to x, the
lateral distance from the trunk. For small x near the trunk, there
will be few leaves, but approaching the boundary of the tree profile,
the leaf density much grow very rapidly. Not only is this an intuitive
model, but is mentioned in [CITATION??]. We suppose that the leaf
density function is of the approximate form ({\bf
  figure~\ref{fig:rho}}) with $\rho_0(z)$ being the maximum leaf
density for a given height.:

\begin{center}
  \begin{equation}
    \rho(z,x) = \dfrac{\rho_0(z)}{1+e^{-6(\frac{2x}{l(z)}-1)}} \label{eqn:rho}
  \end{equation}
\end{center}

\begin{figure}[h!]
  \centering
  \scalegraphics{img/rho.png}
  \caption{Shape of $\rho(z,x)$ for fixed $z$}
  \label{fig:rho}
\end{figure}

Integrating equation~\eqref{eqn:rho} across all values of z and then
doing around the $z$ axis with $\theta$ going from 0 to $2\pi$, we
find the leaf mass with the following expression:

\begin{center}
  \begin{equation}
    \begin{split}
      m &= \int_0^{2\pi} \int_0^{h} \int_0^{l(z)} \rho(z,x) \ dx dz d\theta \\
      &= 2\pi \int_0^{h} \left(\int_0^{l(z)} \dfrac{\rho_0(z)}{1+e^{-6(\frac{2x}{l(z)}-1)}} dx\right) dz \\
      &= 2\pi \int_0^{h} \rho_0(z) \left[ \frac{l(z)}{12} log\left( e^{\frac{12x}{l(z)}} + e^6 \right) \right]_0^{l(z)} dz
    \end{split}
  \end{equation}
\end{center}

We have now defined a leaf mass function which depends on the
parameters:

\begin{enumerate}
\item $l(z)$ The profile function of a tree.
\item $\rho_0(z)$ The maximum leaf density at a height $z$.
\item $h$ The height of the tree.
\end{enumerate}

\subsection{Relationship Between Maximum Leaf Density and Energy}
It has been found experimentally that in some cases the maximum leaf
density for a given height z is directly proportional to the daily
energy observed at that point [CITATION MOTHAFUCKA]. Using notation
described in this paper, we are claiming that $\rho_0(z) \propto
E(z)$. \\

Using this relationship, we can now rewrite our expression for the
leaf mass of a tree substituting $\alpha E(z) = \rho_0(z)$:

\begin{minipage}[l]{\textwidth}
  \begin{center}
    \begin{equation*}
      \begin{split}
        m &= 2\pi\alpha \int_0^{h}
        E(z) \left[ \frac{l(z)}{12} log\left( e^{\frac{12x}{l(z)}} + e^6 \right) \right]_0^{l(z)} dz \\ 
        &= 2\pi\alpha \int_0^{h}
        \left[\int_0^1 I(z,\theta_t)dt\right]
        \left[ \frac{l(z)}{12} log\left( e^{\frac{12x}{l(z)}} + e^6 \right) \right]_0^{l(z)} dz \\
        &= 2\pi\alpha |I_\theta| \int_0^{h}
        \left[\int_0^1 \dfrac{-sin\theta \  l'(z) + cos\theta}{\sqrt{1+(l'(z))^2}} dt\right]  
        \left[ \frac{l(z)}{12} log\left( e^{\frac{12x}{l(z)}} + e^6 \right) \right]_0^{l(z)} dz
      \end{split}
    \end{equation}
  \end{center}
\end{minipage}

\lipsum




\end{document}
