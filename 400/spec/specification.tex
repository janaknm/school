\documentclass[a4paper,12pt]{article}
\usepackage{amsmath}
\usepackage{enumerate}

\pagestyle{empty} \setlength{\parindent}{0mm}
\addtolength{\topmargin}{-0.5in} \setlength{\textheight}{9in}
\addtolength{\textwidth}{1in} \addtolength{\oddsidemargin}{-0.5in}


\begin{document}
\title{Selected Topics in Artificial Intelligence}
\author{Matt Forbes \\ Jianna Zhang (Instructor)}
\date{November 23, 2010}
\maketitle

\section*{Purpose}
To gain a basic understanding of the selected topics in Aritifical
Intelligence. Rather than an in-depth study of a couple subjects, the
plan is to hit a few specific points of some important ideas. This
study should set up a good base for continued learning, most likely in
the graduate level independant study next year.

\section*{Books}

\begin{enumerate}[1)]

\item Russel, S., Norvig, P. (2002). Artificial Intelligence {\it A Modern Approach}.
\item Mitchel, T. M. (1997). Machine Learning.
  
\end{enumerate}

Russel is going to be the main book for this study, using Mitchel as a
supplement for thinner chapters. Neither book goes very in-depth about
genetic algorithms, so hopefully their content will complement each other.

\section*{Material}

\begin{enumerate}[1)]

  \item {\bf Introduction:} Chapter 1 from Russel. Just a quick read-through.

  \item {\bf Intelligent Agents:} Chapter 2 from Russel. Just a quick read-through.
    
  \item {\bf Search and Exploration:} Chapter 4 from Russel. Focus on
    sections 1 and 3, heuristic search strategies and local
    search/optimization problems respectively.

  \item {\bf Knowledge Representation:} Chapter 10 from
    Russel. General ontology, categorizing objects, represenation of
    actions, knowledge and beliefs.

  \item {\bf Uncertainty:} Chapter 13 from Russel. Making rational
    decisions based on the likelihood that the goals will be achieved
    when not all conditions are known.

  \item {\bf Probabilistic Reasoning:} Chapter 14 from Russel, and
    chapter 6 from Mitchel. Capturing and representing uncertain
    knowledge using bayesian networks. Using probabilistic algorithms
    to reason when exact inference is infeasible.

  \item {\bf Statistical Learning Methods:} Chapter 20 from Russel,
    and chapter 4 from Mitchel. Learning probabilistic theories about
    the environment from experience. Methods of learning models, using
    both bayesian and neural networks.

\end{enumerate}

\section*{Projects}
As of right now there aren't any predefined projects for this
study. Projects will be assigned based on the weekly meetings
througout the quarter.

\section*{Additional Comments}

\end{document}

