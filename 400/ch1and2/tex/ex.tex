\documentclass[a4paper,12pt]{article}
\usepackage{amsmath}
\usepackage{enumerate}

\pagestyle{empty} \setlength{\parindent}{0mm}
\addtolength{\topmargin}{-0.5in} \setlength{\textheight}{9in}
\addtolength{\textwidth}{1in} \addtolength{\oddsidemargin}{-0.5in}

\begin{document}

Matt Forbes \\
January 20, 2011 \\
CS 400 - AI Indep. Study \\
Ch 2 \& 4 Discussion

\section*{Intelligent Agents}

\begin{enumerate}[]
\item An agent perceives its environment and is capable of
  action. That is the simple definition, but for an agent to be
  useful, it must have both goals and the 'intelligence' to act in a
  way that its goals are met. In its most basic form an agent should
  be able to sense its environment, perform actions, and detect if a
  goal has been met.
  
\item In the book, there is a simple example of an automatic vacuum
  cleaner that should move around and clean up its environment. One
  property of these cleaners we varied was whether or not it kept
  track of its cleaning history. The vacuum that doesn't track state
  will endlessly roam around its environment even if there is nothing
  to clean.
  
\item When the agent's only goal is a clean room this is fine; if we
  try to minimize the battery spent moving and cleaning it would
  definitely not be a good model. 
  
\end{enumerate}

\subsection*{Agent Applications}

\begin{enumerate}[]
\item A generic model of agents such as this are applicable to a huge
  range of problems. Basically anything that can observe its
  environment and makes decisions from that can be categorized as an
  agent. Automated robots are a general case, while web crawling bots
  are more specific. 
  
\item One of my math professors has told me about his current research
  project, where he is trying to find a chemical dye that when used in
  a solar cell produces the effects he is looking for. The solution to
  this problem is highly experimental, so he generates data by running
  simulations. When he has a set of data, he can analyze which changes
  in variables got him closer to the solution. After making a few
  changes, he runs the simulations again.
  
\item I envisioned this process similarly to an intelligent
  agent. After running the simulation (performing an action), he then
  re-evaluated his environment (perception), finally making changes
  and re-running the simulation (informed action). A* search might
  come in to this equation when searching for a new simulation state
  after receiving the results.
  
\end{enumerate}


\section*{A* and the Traveling Salesman Problem}

\begin{enumerate}[]
\item A* search tries to direct the would-be headless walk through a
  search space. Rather than exhaustively check every possibility, we
  can just expand nodes that we expect to be good bets. To accomplish
  this we need some notion of "good" when describing a node in the
  tree; this is the heuristic. A heuristic function estimates the path
  length to the destination from it. When combined with the path cost
  to the node in question, we have a good way rank nodes.

\item A good definition of the TSP is: "Find a path through a weighted
  graph which starts and ends at the same vertex, includes every other
  vertex exactly once, and minimizes the total cost of the edges."
  Brute forcing this problem would be a disaster, so taking a bit more
  directed approach is more appropriate (but still inefficient.)

\item Minimum spanning trees connect all the vertices of a graph with
  a minimum total edge weight. This total edge weight will always be
  less than or equal to the final path of a TSP tour. If we take the
  TS problem and drop the constraint that you can only visit each node
  once, it would be solved with a MST. Because of this property of
  always giving an optimistic estimation, the MST would be a good
  heuristic for the TSP.

\item So that's how I implemented my solution to the TSP. Using A*
  with MST heuristic to direct an informed search through the large
  search space, my non-optimized implementation can solve a graph with
  about 10 nodes and between 2 and 5 edges per node in a couple
  seconds.

\end{enumerate}

\end{document}



