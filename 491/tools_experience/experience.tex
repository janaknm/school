\documentclass[12pt]{article}

\title{Requirements Specification Tool Experienc}			
\author{Matt Forbes}		
\date{September 28, 2011}					

\begin{document}
\maketitle						

\setcounter{tocdepth}{2}
\pagenumbering{roman} 
\tableofcontents 
\newpage 
\pagenumbering{arabic}

\section{Rmtoo}

\subsection{Steps Taken to Evaluate Software}
\begin{enumerate}
\item Download and install rmtoo archive.
\item Read included Readme.txt and Install.txt.
\item Moved project folder to install directory.
\item Perused the example requirements document/source for rmtoo using
  rmtoo.
\item Copied the project skeleton/template to working directory, and
  added some requirement files. These are simply text files in a very
  simple format.
\item Added a dependent ``topic'' for which requirements can be added.
\item Compiled the project which created an html directory (with a
  fully navigating html site), a clean formatted pdf, and two graphs
  showing dependencies.
\end{enumerate}

\subsection{Initial Thoughts}
Even after reading the included Readme and man files, I wasn't
entirely sure how rmtoo worked. It was made very clear what rmtoo was
capable of, but not necessarily how to go about it. Reading the
example project and makefile solidified my understanding of the
format, which is actually quite simple.

\subsection{Document Generation Process}
Requirements are contained in .req files, which are in a simple
key/value form. Within this file, all the properties and dependencies
of the requirement are listed, such as Name, Description, Priority,
Effort, etc. Topics are specified in .tic files which simple describe
the topic and give it a name to be referred by in the requirements.

Once requirements and topics have been created the provided Makefile
is run, and bingo! All documents are created and ready to view.

\subsection{Analysis}
It is too early in the quarter to tell if this is the right fit for
the job, but as far as simply creating, managing, and linking
requirements, rmtoo does a fine job. 

\section{Axiom}

\subsection{Thoughts}
Axiom appears to have some nice features, but I did not think they
warranted a complicated client/server setup. In order to use Axiom, I
would have had to download and install the server software to my
personal server, download and install the client on a Windows machine
somewhere, and finally start testing. A huge problem with this
software is that it is Windows only, and I do all my work on *nix and 
OS X.

\section{Dia}  

\subsection{Steps Taken to Evaluate Software}
\begin{enumerate}
\item Download and install the dia package.
\item Play around with charts
\end{enumerate}

\subsection{Analysis}
There really isn't a lot of mystery to this application, it really
just does do diagrams. It is not a ``programming'' diagram tool, which
is okay as we are doing a specification, not an implementation. More
than likely there are many hidden features, but on the surface Dia is
very simple and easy to use.

\section{Conclusions}
\begin{enumerate}
\item RMtoo is my first choice for requirements management. It is
  simple, text-based, and does not try to do too much.
\item Dia is a good choice for diagram editing, maybe another student
  will find a more suitable, requirements-oriented application.
\item I highly discourage the use of Axiom as well as any other
  Windows-specific applications. 
\end{enumerate}

\end{document}


