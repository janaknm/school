\documentclass[12pt]{article}

\title{Requirements Specification Tool Research}			
\author{Matt Forbes}		
\date{September 26, 2011}					

\begin{document}
\maketitle						

\setcounter{tocdepth}{2}
\pagenumbering{roman} \tableofcontents \newpage \pagenumbering{arabic}

\section{Rmtoo}

http://www.flonatel.de/index.php?id=9
  
\subsection{Description}
Rmtoo is a text-based command line requirements specification
tool. Rather than being all-inclusive tool-set, it tries to streamline
the process of creating requirements and categorizing them. 

\subsection{Features}
\begin{enumerate}
\item Plain-text file formats allow for easy version control
  integration.
\item Provides a unique way to categorize requirements (``topics'')
\item Supports SCRUM development process
\item Command line access can be much more accessible to programmers.
\end{enumerate}

\subsection{Drawbacks}
\begin{enumerate}
\item No GUI, so there is most likely more of a learning curve than
  other products.
\item Does not include UML or other diagramming tools, project would
  need to be supplemented with an external tool.
\end{enumerate}

\subsection{Thoughts}
I'm personally very comfortable with command line tools, and would
prefer on over a large, slow, bloated GUI. Less time should be spent
tinkering with the requirements specification tool, but rather focus
on quality content.

\subsection{External Documents}
\begin{enumerate}
\item Introduction/Readme: \\
  http://voxel.dl.sourceforge.net/project/rmtoo/v19/rmtooIntroductionV9.pdf
\item Example of generated document: \\
  http://voxel.dl.sourceforge.net/project/rmtoo/v19/requirements.pdf
\end{enumerate}


\section{Axiom}

http://www.iconcur-software.com/solutions.html

\subsection{Description}
Axiom is a client-server type requirements and use case management
tool. An Axiom server can be hosted on either Windows or Linux, but
the client can only be run on Windows. Axiom is a completely GUI-based
program that both manages requirements, and formats a specification.

\subsection{Features}
\begin{enumerate}
\item Users create ``artifacts'' which can be requirements, use cases,
  test cases, or any sort of document needed for the project.
\item Artifacts can be automatically generated based on filters and
  existing requirements.
\item Linking and categorizing of artifacts.
\item Summary and graphical views.
\end{enumerate}


\subsection{Drawbacks}
Seems to be a large hassle to set up properly, the team would need a
server to host the Axiom server software that allows all members to
access it. May have too many features and complications that clutter
real progress on a smaller-scale project.

\subsection{Thoughts}
I would personally rather use a slimmer application since our projects
shouldn't get so large that they are unwieldly. Rmtoo would be a
better choice in my opinion, but this would be fine.

\newpage

\section{Dia}

http://projects.gnome.org/dia/

\subsection{Description}
From the Dia website: ``It can be used to draw many different kinds of
diagrams. It currently has special objects to help draw entity
relationship diagrams, UML diagrams, flowcharts, network diagrams, and
many other diagrams. It is also possible to add support for new shapes
by writing simple XML files, using a subset of SVG to draw the
shape.''

\subsection{Features}
\begin{enumerate}
\item Create many types of diagrams, most importantly UML.
\item Very easy to use and extend.
\end{enumerate}

\subsection{External Documents}
\begin{enumerate}
\item Example diagrams: \\
  http://projects.gnome.org/dia/exempl.html
\end{enumerate}

\subsection{Thoughts}
I have used Dia in the past for UML and object-oriented class design,
and liked the simplicity. While it may not have as many features as
other diagramming tools, it makes up for it with its ease of use.

\end{document}
