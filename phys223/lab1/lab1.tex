\documentclass[a4paper,12pt]{article}

\pagestyle{empty} \setlength{\parindent}{0mm}
\addtolength{\topmargin}{-0.5in} \setlength{\textheight}{9in}
\addtolength{\textwidth}{1in} \addtolength{\oddsidemargin}{-0.5in}

\begin{document}

{\bf Name:} Matt Forbes \\
{\bf Partner:} Stuart  \\ 
{\bf Section:} Phys 233 - Thursday \\
{\bf TA:} Ellis Roe  \\ \\ 
{\bf Lab 1:} The Oscilloscope and its Uses \\

\section{Purpose}
The goal of lab 1 is to become familiar with the oscilloscope and
oscillator; how to adjust the settings and read and interpret output
from the screen. Most exercises in this lab revolved around
experimenting with different settings on both the oscilloscope and the
oscillator, learning how to calculate different parameters such as
frequency and phase shift of the input wave. I would expect our
results to be fairly accurate, with error only prevalent in our
ability to fine tune settings and read measurements from the screen.

\section{Procedure}
In essence, each exercise was introducing a method of measurement for
each property of the oscillator-generated wave, including the
period/frequency, amplitude, phase shift, and frequency ratios. 

\section{Data}
\subsection{Exercise 1}
In the first exercise, we are measured the period and frequency of the
oscillator-generated wave using the equation:
\begin{center}
  $T = ($\# of horizontal cm$) * ($TIME/DIV setting$)$
\end{center}

After setting the oscillator and adjusting the oscilloscope, we simply
measured the horizontal distance of a single period on the
display. Obtaining the period was a simple matter of multiplying by
the TIME/DIV setting. \\ 

\begin{tabular}{|c|c|c|c|c|}
  \hline
  Oscillator Frequency (Hz) & TIME/DIV & Period Length (cm) & Period (s) & Frequency (Hz) \\
  \hline
  20,000 & 10 $\mu s$ & 5 $\pm$ 0.1 & 0.00005 & 20,000 \\
  6,000 & 50 $\mu s$ & 3.4 $\pm$ 0.1 & 0.0017 & 5882 \\
  600 & 0.5 ms & 3.4 $\pm$ 0.1 & 0.0017 & 588 \\
  \hline
\end{tabular}

\subsection{Exercise 2}
Here we measured the peak-to-peak amplitude of the wave, which is the
length from the trough to the crest. Similar to exercise 1, we used a
simple equation to determine the amplitude:
\begin{center}
  $A = ($\# of vertical cm$) * ($VOLTS/DIV setting$)$
\end{center}

We set the oscillator to each frequency, and measured the peak-to-peak
vertical distance on the display. Multiplying by the VOLTS/DIV setting
>gives us the amplitude. \\

\begin{tabular}{|c|c|c|c|}
  \hline
  Oscillator Frequency (Hz) & VOLTS/DIV (V) & Amplitude Length (cm) & Amplitude (V) \\
  \hline
  1,000 & 5 & 6 $\pm$ 0.2 & 30 \\
  500 & 5 & 6 $\pm$ 0.2 & 30 \\
  100,000 & 5 & 6 $\pm$ 0.2 & 30 \\
  \hline
\end{tabular}

\subsection{Exercise 3}
This exercise involved two waves, both generated from the same
oscillator, but one is passed through a ``phase shift network'' which
adjusts the phase shift of an input wave by a specified amount. Our
goal here was to determine the actual shift generated by passing the
wave through the network. We used three separate techniques to do
this.

\subsubsection{Dual-Trace Method}
We first measure the length of a full period in cm, and divide by
360$^{\circ}$ which gives us the phase shift per cm. Next we measure
the horizontal difference between the two waves on the screen and
multiply by the previous ratio, giving us the phase shift. \\

\begin{tabular}{l l}
  Full Phase Length & 6.6 $\pm$ 0.2 cm \\
  Degrees per cm & 54.54$^{\circ}$ \\
  Phase Difference & 1.2 $\pm$ 0.1 cm \\
  Phase Shift & 56.45 \\
\end{tabular}

\subsubsection{X-Y Method}
The X-Y method is much less intuitive than the dual-trace. After
switching the oscilloscope to X-Y mode, the display is a Lissajous
figure depending on relative amplitudes and phase shifts of the two
waves. Using the following equation, we can determine the phase shift:

\begin{center}
  $\theta = sin^{-1}(\frac{A}{B})$
\end{center}

Where A and B are certain dimensions of the Lissajous figure.

\begin{tabular}{l l}
  $A = $ & $5 \pm 0.2$ cm \\
  $B = $ & $6 \pm 0.2$ cm \\
  $\theta = $ & 56.4$^{\circ}$ \\
\end{tabular}


\subsubsection{Sum Method}
Finally, the sum method is used to determine the phase shift. This
technique treats each wave as a vector. Using the sum of the two waves
and a little trigonometry, we have the following equation which can be
solved for our phase shift, $\theta$: 

\begin{center}
  $Sum$ = $A\sqrt{2 + 2\cos\theta}$
\end{center}

Where $Sum$ is equal to peak-to-peak amplitude of the sum of the
original two waves, and $A$ is equal to the peak-to-peak amplitude of
a single original wave. \\

\begin{tabular}{l l}
  $A = $ & $6$ \\
  $Sum = $ & $9.75 \pm 0.4$ cm \\
  $9.75 = $ & $6\sqrt{2 + 2\cos\theta}$ \\
  $\theta = $ & 71$^{\circ}$ \\
\end{tabular} \\

This measurement of $\theta$ is much different than the others, which
is most likely due to the high error in $Sum$, which the screen was
not large enough for us to read without scrolling.

\subsection{Exercise 4}
Having been introduced to the three methods of determining the phase
shift, we now are asked to set the phase angle using the dual-trace
method, and checking with the other two for consistency. \\

\begin{tabular}{|c|c|c|}
  \hline
  $\theta$, Dual-Trace & $\theta$, X-Y & $\theta$, Sum \\
  \hline
  45$^{\circ}$ & 49$^{\circ}$ & 36$^{\circ}$ \\
  \hline
  90$^{\circ}$ & 90$^{\circ}$ & 88$^{\circ}$ \\
  \hline
  105$^{\circ}$ & 68$^{\circ}$ & 106$^{\circ}$ \\
  \hline
\end{tabular} \\

In the first two cases, the X-Y method was very close while the Sum
method was so-so. For some reason on the third case, 105$^{\circ}$, we
could not get the X-Y method to come close. We redid both the
measurements and calculations multiple times to hopefully spot the
reason for the error, but we couldn't. 105$^{\circ}$ is the ratio for
angle per cm we used in our initial dual-trace calculations, which may
be culprit. Either way, our uncertainties in measurements were about
$\pm$ 0.2 cm in both methods. If we could only use one method, I'm not
sure what I would pick, because neither the X-Y or Sum method were
reliable. The dual-trace is probably the most straight forward, so I
would go with that.

\subsection{Exercise 4 (Challenge)}
Our final exercise was to determine the frequency ratios of two
different waves, each generated with its own oscillator, based on
pictures of the Lissajous figure. Based on a helpful hint from Ellis,
we were able to determine them all correctly. They appear in this
order (vertically): $\frac{1}{2}$, $\frac{2}{3}$, $\frac{3}{4}$,
$\frac{3}{5}$, $\frac{4}{5}$, $\frac{5}{6}$.

\section{Analysis}
It seems that the uncertainty in measurements is a fairly large issue
when using an oscilloscope. Our results from exercise 4 show this most
clearly, we measured the same wave using three different methods, and
got different results each time. Usually, the margin of error was
around 0.1 to 0.2 cm which is quite large when you're talking about
waves. 

\section{Conclusion}
This lab was a hands-on way to learn our way around an oscilloscope,
and brought to our attention the amount of uncertainty in its
measuremnts. I am sure this tool will be utilized further in following
labs.

\end{document}
