\documentclass[12pt]{article}
\usepackage{amsmath}
\usepackage{amsfonts}
\usepackage{fancyhdr}
\usepackage[margin=1in,lmargin=0.5in,rmargin=0.5in]{geometry}
\usepackage{indentfirst}
\usepackage{units}

% This allows for multiple columns in one page.
\usepackage{multicol}

\def \TITLE {Lab 4: Sound Waves - Phys223 - Thursday (Ellis Roe)}

\newcommand{\scalegraphics}[1]{%
    \settowidth{\imgwidth}{\includegraphics{#1}}%
    \setlength{\imgwidth}{\minof{\imgwidth}{\columnwidth}}%
    \includegraphics[width=\imgwidth,keepaspectratio]{#1}%
}

\pagestyle{fancy}

\fancyhead{}
\fancyfoot{}

\fancyhead[CO,CE]{Matt Forbes - Lab 4}
\fancyfoot[CO,CE]{Lab 4 - Sound Waves}
\renewcommand{\footrulewidth}{0.4pt}


\lhead{}
\rhead{\thepage}

\begin{document}

\onecolumn
\title{\TITLE}
\author{Matt Forbes, Stuart (partner)}
\maketitle

\twocolumn
\section*{Purpose}
The end goal of each of the exercises in this week's lab was to obtain
an estimate for the speed of sound. Exercise one used an adjustable
tube (changing height of water level) to obtain the resonance
measurements and thus the wavelength of sound. Using the wavelength,
we can find the speed of sound through air. In the second exercise, we
made direct measuremnts of sound using a microphone and an
oscilloscope.

\section*{Procedure}
Exercise one was split in to two parts. First of which was to measure
resonance distances of the speed of sound in a tube of water. We start
with a speaker above the tube generating a wave with constant
frequency. We slowly lower the water level until we here a noticeable
change in the intensity of sound (also noting a spike on the
oscilloscope). We take a measurement of this location and adjust for
spacing between the speaker. Repeating this, we obtain the first few
resonance distances for the tube. Using these values, we can obtain an
average wavelength, and since we know the freqency, the speed of
sound.

In the second part of exercise one, we use the Data Studio software to
measure the FFT of the resulting sound wave after hitting a hollow
plastic tube in front of a microphone. Noting the harmonic frequencies
on the FFT, we can find the resonance locations. We find the speed of
sound in the same way as part one.

Exercise two is broken in to three methods. Each approaches the
calculation of the speed of sound in a different way. Method one uses
the basic chop setting on the oscilloscope to view both input channels
at the same time. We adjust the speaker distance slowly to measure one
full wavelength. Since the frequency is set beforehand, we can
calculate the speed of sound using this wavelength.

Method two uses the x-y setting on the oscilloscope to produce a
Lissajous figure. By adjusting the speaker distance we can change the
shape of the figure, and if we measure the distance between two
identical figures, we get the wavelength. We calculate the speed of
sound in the same way as method 1.

Finally, method 3 directly measures the time a sound pulse takes to
travel a known distance. At this point we hook up a pulse generator to
the oscillator and measure the time difference between pulses on the
screen. We start with the speaker at some fixed point, noting the x
offset, then shift the microphone as far away as possible. Using the
oscilloscope, we measure the time difference between these two
points. Dividing distance by time gives us the velocity.

\section*{Data}

In exercise one, we measure the resonant frequencies experimentally,
but also calculate them theoretically. We list both values here and
show the difference between them. We label $L^P$ the
predicted length and $L^M$ as the measured length. Using
$L^P$ we calculate the wavelength, $\lambda$.

\begin{center}
  \begin{tabular}{|l|l|l|l|}
    \hline
    $L^P$ (cm) & $L^M$ (cm) & $\lambda$ (cm) \\
    \hline
    8.5 & 8.7 $\pm$ 0.1 & 34.8 \\
    25.5 & 25.7 $\pm$ 0.1 & 34.27 \\
    42.5 & 42.8 $\pm$ 0.1 & 34.24 \\
    59.5 & 60.5 $\pm$ 0.1 & 34.57 \\
    \hline
  \end{tabular}
\end{center}

Here we calculate $\bar{\lambda}$ as the average $\lambda$ in the
table above:

\begin{center}
  \begin{equation*}
    \bar{\lambda} = \dfrac{\lambda_1^M + \lambda_2^M + \lambda_3^M + \lambda_4^M}{4} = 34.47 \mathrm{cm}
  \end{equation*}
\end{center}

Using the equation $v = \bar{\lambda}f$ we calculate the speed of
sound using frequency of 1000 Hz:

\begin{center}
  \begin{equation*}
    v_{\mathrm{sound}} = \bar{\lambda} f 
    = (34.7\mathrm{cm})(1000 \mathrm{Hz}) 
    = 344.7 \left(\frac{\mathrm{m}}{\mathrm{s}}\right)
  \end{equation*}
\end{center}

For part two, we obtained the following frequencies for the first
three harmonics: 351 Hz, 718 Hz, and 1006 Hz. Each measurement has an
error factor of $\pm$ 1 Hz. We call the length $L_0$ which is equal to
46.25cm. Below are the calculations used to determine the speed of
sound using each harmonic frequency:

\begin{equation*}
  2L_0 = \lambda_1 = 92.5 \mathrm{cm} 
\end{equation*}
\begin{equation*}
  \begin{split}
    v_{\mathrm{sound}} &= \lambda_1 f_1 \\
    &= (\unit[92.5]{cm})(\unit[351]{Hz}) \\
    &= \unitfrac[324]{m}{s}
  \end{split}
\end{equation*}

\begin{equation*}
  L_0 = \lambda_2 = \unit[46.25]{cm}
\end{equation*}
\begin{equation*}
  \begin{split}
    v_{\mathrm{sound}} &= \lambda_2 f_2 \\
    &= (\unit[46.25]{cm})(\unit[718]{Hz}) \\
    &= \unitfrac[332]{m}{s}
  \end{split}
\end{equation*}

\begin{equation*}
  \nicefrac{2L_0}{3} = \lambda_3 = \unit[30.8]{cm}
\end{equation*}
\begin{equation*}
  \begin{split}
    v_{\mathrm{sound}} &= \lambda_3 f_3 \\
    &= (\unit[30.8]{cm})(\unit[1006]{Hz}) \\
    &= \unitfrac[310]{m}{s}
  \end{split}
\end{equation*}

For exercise two, part one we have the have frequency set to
$\unit[3500]{Hz}$ and we obtain the following measurements for the
wavelength:

\begin{center}
  \begin{tabular}{|l|l|l|}
    \hline
    $x_1$ & $x_2$ & $\lambda$ \\
    \hline
    \unit[55]{cm} & \unit[45]{cm} & \unit[10]{cm} \\
    \unit[45]{cm} & \unit[35]{cm} & \unit[10]{cm} \\
    \unit[35]{cm} & \unit[25]{cm} & \unit[10]{cm} \\
    \hline
  \end{tabular}
\end{center}

Since each $\lambda_i$ is equal to \unit[10]{cm} we can just use
$\bar{\lambda}$ = \unit[10]{cm}, which gives $v_{\mathrm{sound}} =
\bar{\lambda}f = $ \unitfrac[350]{m}{s}.

In the second part of exercise two, we did nearly the same thing, but
instead of matching up the phase, we matched the shape of the
Lissajous figure. Here are the distances for which the figure was the
same, which gives us wavelength:

\begin{center}
  \begin{tabular}{|l|l|l|}
    \hline
    $x_1$ & $x_2$ & $\lambda$ \\
    \hline
    \unit[55]{cm} & \unit[45]{cm} & \unit[10]{cm} \\
    \unit[40]{cm} & \unit[30]{cm} & \unit[10]{cm} \\
    \hline
  \end{tabular}
\end{center}

Which again gives us $v_{\mathrm{sound}} = \bar{\lambda}f = $
\unitfrac[350]{m}{s}. Finally, in the last part of exercise two, we
are measuring the time difference in sound pulses. Here we display the
offset and time at the two locations which allows us to find $\Delta$x
and $\Delta$t, and thus $v_{\mathrm{sound}} = \nicefrac{\Delta
  \mathrm{x}}{\Delta \mathrm{t}}$.

\begin{center}
  \begin{tabular}{|l|l|l|l|l|l|l|}
    \hline
    $x_1$ & $x_2$ & $t_1$ & $t_2$  & $v_{\mathrm{sound}}$ \\
    \hline
    \unit[58]{cm} & \unit[5]{cm} & \unit[0.2]{ms} & \unit[1.6]{ms} & \unitfrac[378.5]{m}{s} \\
    \unit[58]{cm} & \unit[50.5]{cm} & \unit[0.2]{ms} & \unit[0.4]{ms} & \unitfrac[375]{m}{s} \\
    \hline
  \end{tabular}
\end{center}
  
\section*{Analysis}
We calculated the speed of sound many different ways in this lab, some
being much more accurate than others. Exercise one seemed to produce
the most accurate results as it was a combination of many different
resonance lengths. In part two of exercise one, though, not only were
all the estimates for the speed of sound very wrong, but they weren't
even consistently wrong. I would at least partially blame this on the
excess noise in the room we were recording. It was hard to find a
graph of the FFT in Data Studio that did not include a lot of
noise. In a perfectly quiet room, this may not have been a problem. 

In exercise two, we also had quite different results between
subexercises. In addition, all three values were off by quite a bit
from the true value of the speed of sound.

\section*{Conclusion}
Determining the speed of sound is not an easy task, and different
methods return very different results. My preferred method would be
similar to exercise one, where we can fine tune the value across
multiple harmonics..


\end{document}
