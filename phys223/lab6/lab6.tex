\documentclass[12pt]{article}
\usepackage{amsmath}
\usepackage{amsfonts}
\usepackage{fancyhdr}
\usepackage[margin=1in,lmargin=0.5in,rmargin=0.5in]{geometry}
\usepackage{indentfirst}
\usepackage{units}

% This allows for multiple columns in one page.
\usepackage{multicol}

\def \TITLE {Lab 6: Lenses, Mirrors, and Telescopes\\ Phys223 - Thursday (Ellis Roe)}

\newcommand{\scalegraphics}[1]{%
    \settowidth{\imgwidth}{\includegraphics{#1}}%
    \setlength{\imgwidth}{\minof{\imgwidth}{\columnwidth}}%
    \includegraphics[width=\imgwidth,keepaspectratio]{#1}%
}

\pagestyle{fancy}

\fancyhead{}
\fancyfoot{}

\fancyhead[CO,CE]{Matt Forbes - Lab 5}
\fancyfoot[CO,CE]{Lab 6 - Lenses, Mirrors, and Telescopes}
\renewcommand{\footrulewidth}{0.4pt}


\lhead{}
\rhead{\thepage}

\begin{document}

\onecolumn
\title{\TITLE}
\author{Matt Forbes and Stuart(Partner)}
\maketitle

\twocolumn
\section*{Purpose}
In this lab, we examine what happens to rays of light when directed in
to various mirrors and lenses. We start by looking at simple
reflection off of a mirror, then determine the focal length of both
convex and concave lenses. Next, we observe the change in size of an
object when magnified through a lens. Finally, we build a telescope
using two lenses called the ojective and the eyepiece.

\section*{Procedure}
\subsubsection*{Exercise I}
Using a single incident ray from a light source, we measure the
incident and reflected angle whe directed at a flat mirror. We then
repeat the measurements using the primary color rays generated by the
light source. 

Using a concave mirror and five parallel light rays, we determine the
focal length by measuring the distance from the intersection of the
five rays to the center of the mirror. Then, with a compass we
determine the radius of curvature of the mirror and compare. Finally,
we do the same with a convex mirror.

\subsubsection*{Exercise II}
In this exercise, we investigate the difference between concave and
convex lenses in terms of focal length. Again using the five incident
light rays, we measure the focal length of each type of mirror
individually. Next, we nest them together to produce a single lens and
measure the focal length of that. Finally, we slide the lenses apart
to see the effect of light traveling through the two lenses separately
at different places.

\subsubsection*{Exercise III}
We use a 200mm lens to focus the light source of a tree outside the
building on to a screen. When we are able to focus the image, we
measure the distance from the lens and the screen: the image
distance. We repeat with a curved mirror to see the image in front of
the mirror.

Next we set up a lens and screen on the optical bench where we can
shine an image from a light source through the lens and on to the
screen. We then determine a lens position for which the image is in
focus on the screen and measure the size of the image. Moving the
screen along the bench, and finding the two positions of focus, we
create a table of image size and lens distance for each screen
position we choose. Later, we plot these points and calculate the
magnification due to the lens. 

\subsubsection*{Application: Telescope}
In this last section of the lab, we build a telescope using the
provided materials and instructions. Before assembling the telescope,
we measure the focal length of the eyepiece and the objective lens. We
later use these to determine the magnification of the
telescope. Through the telescope, we compare the image size and object
size of a distant poster which has a series of vertical lines running
down it.

\section*{Data}
\section*{Analysis}
\section*{Conclusion}


\end{document}
