\documentclass[12pt]{article}
\usepackage{amsmath}
\usepackage{amsfonts}
\usepackage{fancyhdr}
\usepackage[margin=1in,lmargin=0.5in,rmargin=0.5in]{geometry}
\usepackage{indentfirst}
\usepackage{units}

% This allows for multiple columns in one page.
\usepackage{multicol}

\def \TITLE {Lab 5: Interference and Diffraction\\ Phys223 - Thursday (Ellis Roe)}

\newcommand{\scalegraphics}[1]{%
    \settowidth{\imgwidth}{\includegraphics{#1}}%
    \setlength{\imgwidth}{\minof{\imgwidth}{\columnwidth}}%
    \includegraphics[width=\imgwidth,keepaspectratio]{#1}%
}

\pagestyle{fancy}

\fancyhead{}
\fancyfoot{}

\fancyhead[CO,CE]{Matt Forbes - Lab 5}
\fancyfoot[CO,CE]{Lab 5 - Interference and Diffraction}
\renewcommand{\footrulewidth}{0.4pt}


\lhead{}
\rhead{\thepage}

\begin{document}

\onecolumn
\title{\TITLE}
\author{Matt Forbes and Stuart(Partner)}
\maketitle

\twocolumn
\section*{Purpose}
In this lab, we investigate interference and diffraction in both
double and single slit experiments. We also look at diffraction on CDs
and DVDs to observe the track spacing in the data storage. One of the
main relationships noted in this lab is how the quantities of slit
separation and slit width affect the resulting interference
pattern. We expect that increasing the slit width increases the number
of ``groupings'' of interference extrema. Increasing the slit
separation will decrease the distance between each interference
extrema in a group.

\section*{Procedure}
First, we did a simulated double slit experiment where we used two
identical sinusoidal waves printed on transparent material to
represent light rays from our simulated source. We pinned them 10cm
apart which represented the slit separation. Next we drew a line
perpendicular to the waves 50 cm away from where they were
pinned. This line represents the screen where we can capture the
intensity of the interfering light waves. Lining up the two waves
along the screen, we were able to determine locations of maximum and
minimum inteference (destructive and constructive) and compare them to
the predicted values.

Next, we performed an actual double slit experiment using a 633 nm
laser and slit disk with various slit dimensions. We shine the laser
through several types of double slits and can observe the result on
the screen. Here we see the relationship between slit separation/width
and fringe spacing/grouping. 

Afterwards, we direct the laser beam on a CD to observe the
diffraction pattern on the screen. Using the known wavelength of the
laser and the distance between the CD and screen, we can determine the
track spacing on the CD. We then repeat the experiment with a DVD, and
see that the diffraction pattern is much too large to measure, as it
reaches the ceiling.

Finally, we investigate the diffraction from a single slit
apparatus. Using a similar setup as the double slit, we shine a laser
through single slits of various dimensions. Using measurements of the
minimum on the diffraction pattern, we compare our results with the
expected values from theory.

Further, we determine the thickness of human hair by measuring the
diffraction pattern created when shining a coherent beam (our laser)
at a single human hair. Using similar calculations as the single slit
experiment, we can find the width of this hair.

\section*{Data}

\subsection*{Simulated Double Slit}
For some unknown reason (verified by Ellis), our constructive and
destructive interference locations were not symmetric about the center
point on the screen. This may have been an error in the setup of the
two pinned waves, improperly drawn lines (not completely
perpendicular), or a combination of the two. In any case, here is a
table showing our results:

\begin{center}
  \begin{tabular}{|l|l|l|}
    \hline
    y & Interference Type & $\lambda$ \\
    \hline
    11 $\pm$ 0.25 cm & Constructive & 2.1 cm\\
    4.5 $\pm$ 0.25 cm & Destructive & 0.6 cm\\
    0 cm & Constructive & -\\
    -6.75 $\pm$ 0.25 cm & Destructive & 2.6 cm\\
    -12 $\pm$ 0.25 cm & Constructive & 2.3 cm\\
    \hline
  \end{tabular}
\end{center}

In the above table, we calculated $\lambda$ using the following
formulas derived from equations outlined in the lab text:

\begin{center}
  \begin{equation*}
    \lambda_{\mathrm{max}} = \frac{d}{m} \sin \left( \tan^{-1} \left( \frac{y_{\mathrm{bright}}}{L} \right) \right)
  \end{equation*}
\end{center}

\begin{center}
  \begin{equation*}
    \lambda_{\mathrm{min}} = \frac{d}{m+\frac{1}{2}} \sin \left( \tan^{-1} \left( \frac{y_{\mathrm{dark}}}{L} \right) \right)
  \end{equation*}
\end{center}

\subsection*{Double Slit}
For the double slit experiments, we tried three different dimensions
of slit widths and slit separations. For each combination, we wrote a
description of the interference pattern.

\begin{center}
  \begin{tabular}{|l|l|l|}
    \hline
    Width & Separation & Description \\
    \hline
    0.04 mm & 0.25 mm & Spacing between maxima were \\ & &small but visible. \\
    0.04 mm & 0.5 mm & Same general pattern as with \\ & &0.25 mm separation, but space\\ & &between individual maxima was\\ & &imossible to measure \\
    0.08 mm & 0.25 mm & Same spacing between maxima \\ & &as with 0.04 mm width slits,\\ & &but there were many more\\ & &groupings of maxima in\\ & &this pattern. \\
    \hline
  \end{tabular}
\end{center}

\subsection*{

\section*{Analysis}
In the simulated double slit experiment we saw a larger scale, visual
representation of the interference occuring. It was quite obvious how
two waves hitting the screen in the same position could have varying
amounts of interference, and that maxima and minima were easy to spot.

Actually performing the double slit experiment with the laser was
instructive as we were able to see the impact that changing the two
slit parameters (slit width and slit separation) had on the resulting
interference pattern. Also experimenting with a CD was neat, as they
were always a mystery.

Finally, the single slit experiment showed pure diffraction and how
the intensities change at different locations on the screen. Also the
slit size had a very large impact on the separation of the intensity
maxima and minima.

\section*{Conclusion}
This lab gave the double slit experiment a tangible explanation and
clearly showed how and why light interference happens. Not only did we
see diffraction/intereference patterns through slits, but we also saw
a little bit about how digital storage works (CDs and DVDs). 

\end{document}
