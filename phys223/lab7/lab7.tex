\documentclass[12pt]{article}
\usepackage{amsmath}
\usepackage{amsfonts}
\usepackage{fancyhdr}
\usepackage[margin=1in,lmargin=0.5in,rmargin=0.5in]{geometry}
\usepackage{indentfirst}
\usepackage{units}

% This allows for multiple columns in one page.
\usepackage{multicol}

\def \TITLE {Lab 7: Refraction\\ Phys223 - Thursday (Ellis Roe)}

\newcommand{\scalegraphics}[1]{%
    \settowidth{\imgwidth}{\includegraphics{#1}}%
    \setlength{\imgwidth}{\minof{\imgwidth}{\columnwidth}}%
    \includegraphics[width=\imgwidth,keepaspectratio]{#1}%
}

\pagestyle{fancy}

\fancyhead{}
\fancyfoot{}

\fancyhead[CO,CE]{Matt Forbes - Lab 7}
\fancyfoot[CO,CE]{Lab 7 - Refraction}
\renewcommand{\footrulewidth}{0.4pt}


\lhead{}
\rhead{\thepage}

\begin{document}

\onecolumn
\title{\TITLE}
\author{Matt Forbes and Stuart(Partner)}
\maketitle

\twocolumn
\section*{Purpose}
In this week's lab, we observe the behavior of light traveling between
mediums which have a different ``index of refraction.'' This index of
refraction is a relation of how fast light can travel through a given
medium. We first measure the effects of refraction by comparing the
angles of incidence and refraction. Then we are able to produce total
internal reflection, and determine the critical angle for which this
phenomenon occurs. Next, we use Snell's law to find the index of
refraction for both water and oil. And finally, we calculate the focal
length of a concave lens in two ways, one of which is the lens makers
formula.

\section*{Procedure}
\subsubsection*{Exercise I: Prism}
We shine white light through a prism (in our case, rhombus) in such a
way that the refracted light has been separated in to individual
colors. Noting that light with a larger frequency has a larger index
of refraction for a given material, we are able to predict and verify
which color will refract the most.

\subsubsection*{Exercise I: Slab}
Using the same rhombus, but using two parallel surfaces this time, we
calculate angle of incidence and refraction of an incident light
ray. Making multiple measurements for different angles of incidence,
we are able to calculate the index of refraction of the acrylic
rhombus by averaging Snell's law.

\subsubsection*{Exercise II: Total Internal Reflection}
In this exercise, we experimentally find the critical angle of the
rhombus by finding the angle of incidence which results in only
reflection at the surface; that is, no light passes through the
incident surface on the rhombus. Using Snell's law, we can determine
the theoretical critical angle by setting $\theta_2$ to 90$^{\circ}$
and solving for $\theta_1$.

\subsubsection*{Exercise III: Application}
Here we measure the index of refraction of two substances contained in
prisms. To do this, we shine a laser pointer through the prism,
marking it's entrance and exit location. Then we find the angle
between the incident ray and the refracted ray. With this angle, we
can calculate the index of refraction using a given formula.

\subsubsection*{Exercise IV: The Lens Makers Formula}
We shine five parallel rays at a double concave lens, and determine
the focal length by finding the distance from the intersection of the
reflected rays and the lens itself. Then, knowing the radius of
curvature of the lens, we use the Lens Makers formula to find a second
value for focal length. 

\section*{Data/Analysis}
\subsubsection*{Exercise I: Prism}
Observing the refracted rays from the prism, we see the following
colors: blue, green, yellow, orange, and red. Using Snell's law, we
see that:
\[\theta_\mathrm{r} \approx \sin \theta_\mathrm{r} = \frac{n_1\sin\theta_\mathrm{i}}{n_2}\]    

Here, $n_2$ is 1, air, and $n_1$ is the index of refraction of the
prism. We know that $n_2$ is proportional to the frequency of light,
and since blue light has the largest frequency, we'd expect it to have
the most refraction. This is what we see.

Shining the colored rays through the rhombus, we see that the
refracted rays intersect, and thus are not parallel. Justification of
this can be seen from Snell's law.

\subsubsection*{Exercise II: Slab}
For the following three angles of incidence, we measure the angle of
refraction and thus index of refraction. 

\begin{center}
  \begin{tabular}{|l|l|l|}
    \hline
    $\theta_i$ & $\theta_r$ & $n_2$ \\
    \hline
    18.0 $\pm$ 1 & 10.5 $\pm$ 1 & 1.69 \\
    40.5 $\pm$ 1 & 26.5 $\pm$ 1 & 1.46 \\
    26.0 $\pm$ 1 & 17.5 $\pm$ 1 & 1.45 \\
    \hline
  \end{tabular}
\end{center}

When we average the index of refraction for each of these
measurements, we get $\bar{n} = 1.53$. Compared to the actual value of
1.5, we say this is a pretty good method.

\subsubsection*{Exercise II: Total Internal Reflection}
Using the angular part of the rhombus, we determined the minimum angle
so that total internal reflection is achieved. Then, measuring the
entrance and exit of the ray, we can measure the incident and
reflected angle. Our measurements give: 

\begin{center}
  \begin{tabular}{l l l}
    $\theta_c$ &= & 44$^{\circ} \pm$ 4$^{\circ}$ \\
    $\theta_r$ &= & 40$^{\circ} \pm$ 4$^{\circ}$ \\
  \end{tabular}
\end{center}


We would expect these two angles to be equal to each other, because
this is just simple reflection as in mirrors. We can calculate the
theoretical critical angle using Snell's law as follows:

\[\theta_c = \sin^{-1}\left(\frac{2}{3}\right) = 41.8^{\circ}\]

The percent difference in the experimental and theoretical values for
the critical angle is 5\%. 

We observe the brightness of the internally reflected ray increase as
the incident angle is increased past the critical angle. We know that
the critical angle is inversely proportional to the index of
refraction. n will be smaller for violet light, thus the critical
angle will be greater.

\subsubsection*{Exercise III: Application}
In this exercise, we need te measure the angle, $\delta$, which is how
much the refracted ray deviates from the incident angle in the
prism. Using this angle and a provided geometric formula, we can
determine the index of refraction for the substance:

\begin{center}
  \begin{equation*}
    n = \dfrac{\sin\left[\frac{1}{2}(\theta + \delta)\right]}{\sin\left[\frac{1}{2}(\theta)\right]}
  \end{equation*}
\end{center}

For water, we measure $\delta = 22.5^{\circ} \pm 1^{\circ}$ using the
procedure described above. Simply plugging in $\delta$ in to the
formula, we find that $n = 1.39$ which is quite close to the known
value of 1.33.

For the oil substance, we similarly measure $\delta = 38^{\circ} \pm
1^{\circ}$ and find $n = 1.51$.

\subsubsection*{Exercise IV: The Lens Makers Formula}
First we measure the radius of convergence for the double concave lens
as 13.6 cm and focal length of 6.8 cm using the technique from lab
6. Using $R_1 = R_2 = -6.8cm$ and $n = 1.5$ for the lens, we calculate
the focal length with the Lens Makers formula:

\[ \frac{1}{f} = (n-1)\left(\frac{1}{R_1}+\frac{1}{R_2}\right) \]

\noindent Solving for $f$ and plugging in our measurements:

\begin{align*}
  f &= \dfrac{1}{(n-1)\left(\frac{1}{R_1} + \frac{1}{R_2}\right)}\\
  &= \dfrac{1}{(\frac{1}{2})\left(\frac{2}{-6.8}\right)} \\
  &= -6.8 \mathrm{cm}
\end{align*}

So the focal length of a concave lens is negative, as we found using
the Lens Makers formula. The thickness of the lens could possibly
change the radius of curvature, and thus the focal length.

\section*{Conclusion}
We thoroughly investigated properties of light traveling between
mediums and how different materials can change the angle for which
light refracts. Throughout, we find Snell's law to be a very useful
tool in predicting how light will react to an interface between two
mediums. Also, we determine the conditions for refraction, the minimum
angle called the critical angle. Finally, we find a new way to
determine focal length using the index of refraction and radius of
curvature of a lens.

\end{document}
