\documentclass[a4paper,12pt]{article}
\usepackage{amsmath}
\usepackage{enumerate}

\pagestyle{empty} \setlength{\parindent}{0mm}
\addtolength{\topmargin}{-0.5in} \setlength{\textheight}{9in}
\addtolength{\textwidth}{1in} \addtolength{\oddsidemargin}{-0.5in}


\begin{document}
\title{Matt Forbes}
\author{Assignment 6}
\date{March 11, 2011}
\maketitle

\section*{6.2}
All peers might act like both the server and the client at the same
time. After obtaining some list of seeders, a client might try and
make connections with all/some of those addresses. When a connection
is established, the client could request different parts of the file
it needs to download. While all of this is happening, the client will
also be acting like a server by listening for connections. This time,
other peers connect to it and request parts of the file, and are
uploaded when found. 

\section*{6.19}
Assuming the book means the process ID given to an program by the OS,
that would be a pretty bad way of identifying services. It would
completely defeat the purpose of ``well-known'' ports as process IDs
not only differ from host to host, but also change when a program is
restarted. To get around this, all connections would need to first be
filtered through some auxilary service like inetd that can forward a
request along to the correct service.

\section*{7.3}
DNS servers map domain and subdomain names to actual IP
addresses. Losing that ability would cripple the net for most
individuals, seeing as not many people know the exact IP address for
any of the web sites or email addresses they use. It would render most
of the world internet-less until the DNS servers were put back online
(assuming their data is all backed up and locked away).

\section*{7.6}
If a DNS server in charge of returning the IP address for such a
machine, it may be updating dynamically based on server load. Another
way this could be achieved is if multiple IPs point to a machine but
the domain name is just associated with one of them.

\section*{7.11}
Every 24 bits are split in to four groups of 6 bits. Each individual
group is then encoded in to an ASCII character. 4 ASCII characters is
32 bits, so we gain 8 bits for every 24. 4560 bytes is 36480 bits, or
1520 24-bit groups. Each 24-bit group is transformed in to a 32-bit
group and thus the file size grows to 48640 bits, or 6080
bytes. Adding a CR+LF (2 bytes) after 110 consecutive bytes means an
extra 110 bytes for a grand total of 6190 bytes.

\section*{7.13}
The MP3 file being sent could be split in to four pieces of 1MB each
(the limit his friend's ISP set). He could then send each of these
four parts in order, using the content-id of the email header to
specify the order. Once his friend has recieved the four parts, he
could rig them back together and have his file.

\section*{7.15}
White space generally means the ``space'' character, but whether
that's just one or multiple isn't necessarily defined. The space
character isn't the only thing that statisfies ``blank space.'' In
html the string $&nbsp;$ is rendered as white space. New line's are
rendered as space. Sometimes non-printables can be rendered as spaces.

\section*{7.18}
Of course not. The IMAP interface must be the same across all
implementations so that a single client can make the same request to
any and recieve the same response. An actual implementation should be
free to design the actual mailbox structure however they want, granted
they know how to respond to client IMAP requests.

\section*{7.23}
With just the information given in the question, I don't really know
why a domain name shouldn't end with a digit. Any browser or program
that uses URLs should be able to determine whether or not it's working
with an IP or a DNS name. Subdomains can end in digits, I'm not sure
if the question is asking specifically about the last character of the
TLD or something else.

\section*{7.44}
Assuming equal distribution across the month, one tenth of their
customers will watch a movie at some time on any given day (3 movies
per 30 days, $\frac{3}{30} = \frac{1}{10}$). One tenth of their
customers is 5,000, and two thirds of those will be watched at
9pm. Rounding up, that's 556 movies each requiring 6Mbps for a total
of 3333Mbps served. OC-12 lines have up to 622Mbps capability. The
video server will need is 5-6 such connections.

\section*{7.47}
I don't want to look up 25 random websites that are most likely all
domain-squatting spam. My estimate is that at least 80\% of the sites
I might type in from those categories would be a spam page.

\section*{8.1}


\section*{8.3}

\section*{8.30}

\section*{8.41}

\section*{private key crypto systems (10)}

\section*{explain public key crypto systems (10)}

\end{document}
