\documentclass[a4paper,12pt]{article}
\usepackage{amsmath}
\usepackage{enumerate}

\pagestyle{empty} \setlength{\parindent}{0mm}
\addtolength{\topmargin}{-0.5in} \setlength{\textheight}{9in}
\addtolength{\textwidth}{1in} \addtolength{\oddsidemargin}{-0.5in}


\begin{document}
\title{Matt Forbes}
\author{Assignment 5}
\date{March 4, 2011}
\maketitle

\section{Server Documentation}
My server implements all elements declared in the RFA protocol. It
passes both my tests and tests written by Dr. Nelson. The server
relies on the functions defined in netlib.h, which along with
netlib.c, should be included in the directory. There are a few ways
that the code could be cleaned up, but I'd rather not break it in the
process.

\subsection{Overview}
\begin{enumerate}[]
  \item The bulk of the program is in the function handle\_client,
    which is called by a forked process after a client connection is
    made. In this function, an array of all the open file descriptors
    is held. Rather than handing back actual file descriptors to the
    client, indexes to this array are passed. I'm not sure why I did
    this.

  \item When a client makes a request, an action is performed based
    on the first character of the message. Currently, the request
    string is parsed character by character which is something that
    could be cleaned up. 

\end{enumerate}

\subsection{Structures}
Unfortunately, I didn't use any interesting data structures except for
a linked list. It's main usage was in loading up the authorized
hostnames; it made it easy to just read the .rfahosts line-by-line and
throw it in to the list.

\subsection{Tests}
Unlike my client code, I didn't do much testing until the program was
complete. After getting the Open procedure working and parsing down,
the rest of the functionality just fell in to place. Besides running
the 'try' test script, I played around with ncat to make sure files
were being written and read correctly.

\subsection{Possible Improvements}
One thing that really bothers me about my code is how request strings
are being parsed. It works just fine, but the same code is duplicated
on multiple lines in multiple functions. My initial attempt was to use
the strtok function from the string library, but that breaks when
filenames have spaces. In hindsight, I should have used that method
for all the requests except for Open, which is a special case.

\section{Client Documentation}


\end{document}
