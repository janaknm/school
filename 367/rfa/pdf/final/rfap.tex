\documentclass[a4paper,12pt]{article}
\usepackage{amsmath}
\usepackage{enumerate}

\pagestyle{empty} \setlength{\parindent}{0mm}
\addtolength{\topmargin}{-0.5in} \setlength{\textheight}{9in}
\addtolength{\textwidth}{1in} \addtolength{\oddsidemargin}{-0.5in}


\begin{document}
\title{Matt Forbes}
\author{Assignment 5}
\date{March 4, 2011}
\maketitle

\section{Server Documentation}
My server implements all elements declared in the RFA protocol. It
passes both my tests and tests written by Dr. Nelson. The server
relies on the functions defined in netlib.h, which along with
netlib.c, should be included in the directory. There are a few ways
that the code could be cleaned up, but I'd rather not break it in the
process.

\subsection{Overview}
\begin{enumerate}[]
  \item The bulk of the program is in the function handle\_client,
    which is called by a forked process after a client connection is
    made. In this function, an array of all the open file descriptors
    is held. Rather than handing back actual file descriptors to the
    client, indexes to this array are passed. I'm not sure why I did
    this.

  \item When a client makes a request, an action is performed based
    on the first character of the message. Currently, the request
    string is parsed character by character which is something that
    could be cleaned up. 

\end{enumerate}

\subsection{Structures}
Unfortunately, I didn't use any interesting data structures except for
a linked list. It's main usage was in loading up the authorized
hostnames; it made it easy to just read the .rfahosts line-by-line and
throw it in to the list.

\subsection{Tests}
Unlike my client code, I didn't do much testing until the program was
complete. After getting the Open procedure working and parsing down,
the rest of the functionality just fell in to place. Besides running
the 'try' test script, I played around with ncat to make sure files
were being written and read correctly.

\subsection{Possible Improvements}
One thing that really bothers me about my code is how request strings
are being parsed. It works just fine, but the same code is duplicated
on multiple lines in multiple functions. My initial attempt was to use
the strtok function from the string library, but that breaks when
filenames have spaces. In hindsight, I should have used that method
for all the requests except for Open, which is a special case.

\section{Client Documentation}
I took a much different approach to the client library. Rather than
blindly forge ahead and code out the whole project, I spent some time
thinking about how the parts would fit together beforehand. Coupled
with unit testing, this program was smooth sailing. As far as
functianality goes, it passed all my unit tests, and Dr. Nelson's test
scripts. All seems well.

\subsection{Overview}
As the assignment entails, there are really just four functions that
need to be provided: open, read, write, and close -- the stand file
manipulation routines. They are prototyped in a way that they
seemlessly mask the corresponding system calls and provide drop-in
network availability.

\subsection{Structures}
The three structures I built were as follows:

\begin{enumerate}[1)]
  \item {\bf host:} Hold server information so that connections can be
    reused. They are uniquely identified by their hostname/port
    combination. An array of these are stored globally and are checked
    for existence when a server connection is requested. Multiple
    files can be opened with just a single connection to an RFA server.

  \item {\bf file\_d:} I'm keeping an array of all the remote file
    descriptors opened, each represented as a {\bf file\_d}. When a
    remote file is opened, the first open index in this global array
    of remote file descriptors is added to 1,000,000 in order to
    guarantee an invalid local file descriptor. Indexes to this array
    can be found by simply subtracting 1,000,000 from the file
    desriptor. {\bf file\_d}s hold a pointer to the corresponding
    host, so it's easy to just grab the corresponding socket.
    
  \item {\bf pathinfo:} Pathnames are specified in the format:
    [[port@]machine:]path. A {\bf pathinfo} breaks this string in to
    the associated pieces, making it much easier to decide how to
    handle the opening of the file. The decision to use a structure
    for this task was motivated by my less-than-optimal solution in
    the server.

\end{enumerate}

\subsection{Tests}
Knowing that small mistakes can completely throw things off
(especially in C), I decided to incrementally implement each function
with a ``suite'' of unit tests. Compared to some of the really nice
unit test frameworks out there, mine are really just quick checks
through printf. My goal was to represent all possible use cases of
each function print out the results, which I just compared by hand as
there weren't too many. In a larger project, I'd definitely want to be
able to automatically detect if tests were passed/failed rather than
just inspecting them manually. Developing this way was much more
stress-free, when I made a change to the program, I could be
comfortable that nothing broke.


\end{document}
