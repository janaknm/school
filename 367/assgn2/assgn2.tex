\documentclass[a4paper,12pt]{article}
\usepackage{amsmath}
\usepackage{enumerate}

\pagestyle{empty} \setlength{\parindent}{0mm}
\addtolength{\topmargin}{-0.5in} \setlength{\textheight}{9in}
\addtolength{\textwidth}{1in} \addtolength{\oddsidemargin}{-0.5in}


\begin{document}
\title{Assignment 2}
\author{Matt Forbes}
\date{January 21, 2011}
\maketitle

\section*{2.6}
A binary signal sent over a 3kHz channel with SNR of 20dB has a
maximum achievable data rate of $3000\log_2(1+100) = 19974$ bits/sec.
\section*{2.10}
Frequency is $\dfrac{c}{\lambda}$ where $c$ is the speed of light. So
with a wavelength of $1cm = 0.01m$, then $f = \dfrac{3*10^8}{0.01} =
3*10^{10}$. The lower bound is $f = \dfrac{3*10^8}{5} = 6*10^7$.
\section*{2.11}
If the beam was going to miss the detector, then its center would be
1mm above the center of the detector. The detector is 100m away from
the origin of the beam, and in this situation creates a triangle with
a 100m side and a 0.001m side. The angle between them is
$tan^{-1}\dfrac{0.001}{100} = 5.73*10^{-4}$ degrees.
\section*{2.13}
Round-trip delay time for GEO satellites is about 270ms, for MEO satellits
it's 35-85ms, and for LEO it's just 1-7ms.
\section*{2.14}
A call from the north pole to the south has to cover the distance of
exactly half the circumference of Earth. Assuming the radius is
$6371km + 750km = 7031km$ and a perfect circle, half the circumference
is $\pi7031 = 22088km$. It would take light $\dfrac{22088}{1000000} =
0.022s$ = 22ms. There is about 4ms of latency for the trip up to the
satellites and back down, and require six switches between satellites,
which is another 60 microseconds = 0.06ms. So the total latency of a
call would be $22 + 4 + 0.06 = 26.06ms$.

\end{document}
