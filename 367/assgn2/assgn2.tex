\documentclass[a4paper,12pt]{article}
\usepackage{amsmath}
\usepackage{enumerate}

\pagestyle{empty} \setlength{\parindent}{0mm}
\addtolength{\topmargin}{-0.5in} \setlength{\textheight}{9in}
\addtolength{\textwidth}{1in} \addtolength{\oddsidemargin}{-0.5in}


\begin{document}
\title{Assignment 2}
\author{Matt Forbes}
\date{January 21, 2011}
\maketitle

\section*{2.6}
A binary signal sent over a 3kHz channel with SNR of 20dB has a
maximum achievable data rate of $3000\log_2(1+100) = 19974$ bits/sec.
\section*{2.10}
Frequency is $\dfrac{c}{\lambda}$ where $c$ is the speed of light. So
with a wavelength of $1cm = 0.01m$, then $f = \dfrac{3*10^8}{0.01} =
3*10^{10}$. The lower bound is $f = \dfrac{3*10^8}{5} = 6*10^7$.
\section*{2.11}
If the beam was going to miss the detector, then its center would be
1mm above the center of the detector. The detector is 100m away from
the origin of the beam, and in this situation creates a triangle with
a 100m side and a 0.001m side. The angle between them is
$tan^{-1}\dfrac{0.001}{100} = 5.73*10^{-4}$ degrees.
\section*{2.13}
Calculate end-to-end transit time for a packet for both GEO (35800km)
MEO (18000km) and LEO (750km).
\section*{2.14}

\end{document}
